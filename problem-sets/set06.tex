\documentclass{homework}
\course{Math 5522H}
\author{Jim Fowler}
\input{preamble}

\begin{document}
\maketitle

\begin{inspiration}
  \textit{Logic sometimes makes monsters.} For half a century we have
  seen a mass of bizarre functions which appear to be forced to
  resemble as little as possible honest functions which serve some
  purpose. More of continuity, or less of continuity, more
  derivatives, and so forth.
  \byline{Henri Poincar\'e} % can someone find the actual reference for this?!
\end{inspiration}

\section{Terminology}

\begin{problem}
  What is meant by uniform convergence on compact sets?
\end{problem}

\begin{problem}
  What is a biholomorphism $f : U \to V$?
\end{problem}

\begin{problem}
  Define \textbf{star-convex}.
\end{problem}

\section{Numericals}

\begin{problem}\label{max-modulus-numerical}
  Consider $f(z) = 4z - (2z - i)^2 - 2i - 4$.  Find $z$ in 
  \[
    \{ x + iy \in \C : 0 \leq x \leq 1 \mbox{ and } 0 \leq y \leq 1 \}
  \]
  maximizing $\abs{f(z)}$.
\end{problem}

\section{Exploration}

\begin{problem}
  Sometimes things are \textit{much} better in the complex case than
  the real case.  Find a sequence of smooth functions
  $f_n : [-1,1] \to \R$ converging uniformly to $f(x) = \abs{x}$ which
  is not differentiable at zero.  Can the uniform limit of holomorphic
  functions fail to be holomorphic?
\end{problem}

\begin{problem}\label{hadamard-three-lines}Prove \textbf{Hadamard's
    three-lines theorem} which is about the strip
  \[
    S_{a,b} := \{ x+iy \in \C : a \leq x \leq b\}
  \]
  bounded by two horizontal lines, i.e., the line with real part $a$
  and the line with real part $b$.  Suppose $f : S_{a,b} \to \C$ is a
  bounded continuous function which is holomorphic on the interior of
  $S_{a,b}$.  Define $M : [a,b] \to \R$ by
  \[
    M(x) := \sup_{y \in \R} \abs{f(x+iy)},
  \]
  that is, $M(x)$ is the supremum of $\abs{f}$ on the (third!) line with real part $x$.
  
  Show that, if $t \in [0,1]$, then \[
    M\left( ta + (1-t)b \right) \leq M(a)^t M(b)^{1-t}.
  \]
\end{problem}

\begin{problem}\label{schwarz-lemma}Suppose $f : B_1(0) \to B_1(0)$ is holomorphic and $f(0) = 0$.  Show
  that $\abs{f(z)} \leq \abs{z}$ for all $z \in B_1(0)$.  How large
  could $\abs{f'(0)}$ be?  This is the \textbf{Schwarz lemma}.
\end{problem}

\begin{problem}\label{schwarz-lemma-2}Suppose again that $f : B_1(0) \to B_1(0)$ is holomorphic, and
  suppose further that $\abs{f(z)} = \abs{z}$ for some nonzero
  $z \in \C$.  Show that $f$ must be a rotation, i.e., there is some
  $\lambda \in \C$ with $\abs{\lambda} = 1$ and $f(z) = \lambda z$.
  (This is also the Schwarz lemma.)
\end{problem}

\begin{problem}\label{schwarz-lemma-3}Suppose that
  $f : B_1(0) \to B_1(0)$ is holomorphic and $\abs{f'(0)} = 1$.  What
  can you deduce about $f$?
\end{problem}

\begin{problem}\label{schwarz-reflection-principle-2}Consider the
  closed upper half-plane
  \[
    H := \{ x + iy \in \C : x \in \R \mbox{ and } y \geq 0 \}.
  \]
  Suppose $f : H \to \C$ is continuous, holomorphic on the interior of
  $H$, and sends reals to reals, i.e., if $x \in \R$ then
  $f(x) \in \R$.  Use a trick like \ref{schwarz-reflection-principle}
  to describe an entire function agreeing with $f$ on its domain.
\end{problem}

\begin{problem}(
  Consider the closed disk
  \[
    D := \{ z \in \C : \abs{z} \leq 1 \},
  \]
  and suppose $f : D \to D$ is continuous, holomorphic on the interior
  of $D$, sends boundary to boundary, i.e., if $z \in \partial D$ then
  $f(z) \in \partial D$, and misses an interior point, i.e., there is
  some $w$ in the interior of $D$ which is not in the image of $f$.

  Combine \ref{cayley-transform} and
  \ref{schwarz-reflection-principle-2} to produce an entire function.
  What can you deduce about $f$?
\end{problem}

\section{Prove or Disprove and Salvage if Possible}

\begin{problem}\label{cauchy-for-starlike}Suppose $U \subset \C$ is a
  star-convex open set, and $f : U \to \C$ is holomorphic, and
  $\gamma : [0,1] \to U$ is a smooth closed curve.  Then
  \[
    \int_\gamma f(z) \, dz = 0.
  \]
\end{problem}

\begin{problem}\label{moreras-theorem}If a function $f : B_r(0) \to \C$
  satisfies % missing continuous!
  \[
    \int_\gamma f(z) \, dz = 0
  \]
  for all piecewise smooth closed curves $\gamma$ in the disk
  $B_r(0)$, then $f$ is holomorphic.
\end{problem}

\begin{problem}\label{uniform-convergence-holomorphic}If the sequence of
  holomorphic functions $f_n : U \to \C$ converge pointwise to
  $f : U \to \C$, then $f$ is holomorphic.
\end{problem}

\begin{problem}\label{automorphisms-of-disk}Suppose $f, g : B_r(0) \to B_r(0)$ are holomorphic and
  $f \circ g = g \circ f$ are the identity on $B_r(0)$.  Then there is
  $w \in B_r(0)$ and $\theta \in [0,2\pi)$ with
  \[
    f(z) = e^{i\theta} \cdot \frac{w - z}{1 - \conj{w}z}.
  \]
  This should remind you of \ref{blaschke-factors}.  This result
  highlights the rigidity of biholomorphic functions as compared to,
  say, homeomorphisms.
\end{problem}

\end{document}
