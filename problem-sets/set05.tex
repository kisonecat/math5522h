\documentclass{homework}
\course{Math 5522H}
\author{Jim Fowler}
\input{preamble}

\begin{document}
\maketitle

\begin{inspiration}
  On April 9, 1975, Congressman Robert Michel brandished a list of new
  NSF grants on the floor of the House of Representatives and selected
  a few that he thought might represent a waste of the taxpayers'
  money. One of them \ldots was called ``Studies in Complex
  Analysis.'' Michel's comment was, `` `Simple Analysis' would,
  hopefully, be cheaper.''
  \byline{Gerald B. Folland} % the American Mathematical Monthly (vol 780, Oct 1998, pg. 780)
\end{inspiration}

\section{Terminology}

\begin{problem}
  What is a \textbf{simply-connected} domain?
\end{problem}

\begin{problem}
  What does it mean to say that two curves $\alpha$ and $\beta$ are
  \textbf{homologous}?
\end{problem}

\begin{problem}
  What is a \textbf{null homotopic} path?
\end{problem}

\begin{problem}
  Given a function $f : U \to \C$, what is meant by a \textbf{branch} of the log of $f$?
\end{problem}

\section{Numericals}

\begin{problem}
  Let $f(z) = (z^2-z-1)/z^3$, and consider the curve
  $\gamma : [0,2\pi] \to \C$ by $\gamma(\theta) = f(e^{i\theta})$.
  Compute the winding number $n(\gamma,0)$.
\end{problem}

\begin{problem}
  Evaluate the \textbf{Dirichlet integral}
  \[
    \int_0^\infty \frac{\sin x}{x} \, dx.
  \]
\end{problem}

\begin{problem}
  Define $\gamma : [0,2\pi] \to \C$ by $\gamma(\theta) = e^{i\theta}$.
  Suppose $p \in \C[z]$ is a degree $n$ polynomial with distinct roots
  $z_1,\ldots,z_n$ in the unit disk, and evaluate
  \[
    \int_\gamma \frac{z^m \, p'(z)}{p(z)} \, dz.
  \]
\end{problem}

\begin{problem}
  Again define $\gamma : [0,2\pi] \to \C$ by
  $\gamma(\theta) = e^{i\theta}$, and then evaluate
  \[
    \int_\gamma \frac{\conj{w}}{w - z} \, dw
  \] for $z \in \C$ with $\abs{z} < 1$.
\end{problem}

\begin{problem}
  Consider subsets of the real line $A = (-\infty,-1] \cup [1,\infty)$
  and $B = [-1,1]$.  Is it possible to define a branch of the
  logarithm of $z^2 - 1$ on $\C \setminus A$?  What about on
  $\C \setminus B$?
\end{problem}

\section{Exploration}

\begin{problem}
  If we can find a branch of log $f$, then we can define a
  single-valued $\sqrt{f(z)}$ via $e^{(1/2) \, \log f}$.  Does the
  converse hold?  Find an open set $U \subset \C$ and a holomorphic
  function $f : U \to \C$ so that there is \textit{no} branch of log
  of $f$, but it is nevertheless possible to define a single-valued
  $\sqrt{f(z)}$.
\end{problem}

\begin{problem}
  Liouville's theorem (\ref{liouville-theorem}) states that a
  nonconstant holomorphic function $f : \C \to \C$ is not bounded.
  Seeing \ref{entire-is-dense}, we might ask how ``small'' can the
  image of holomorphic function be.  Find a holomorphic function
  $f : \C \to \C$ with image equal to $\C \setminus \{w\}$.
  (Incidentally, by \textbf{Picard's little theorem}, you will have
  trouble finding an entire function with image missing two points.)
\end{problem}

\begin{problem}\label{cauchy-inequalities-2}Suppose
  $f(z) = \sum_{n=0}^\infty a_n z^n$ is holomorphic in the disk
  $B_R(0)$.  If $0 < r < R$, show that
  \[
    \abs{a_n} \leq \frac{1}{r^n} \displaystyle\sup_{z \in \partial B_r(0)} \abs{f(z)}.
  \]
  This extends \ref{cauchy-inequalities}.
\end{problem}

\begin{problem}
  Consider a nonconstant polynomial $p \in \mathbb{C}[z]$ of degree $n$.  Compute
  \[
    \lim_{r \to \infty} \int_{\partial B_r(0)} \left( \frac{p'(z)}{p(z)} - \frac{n}{z} \right) dz.
  \]
  Can you perform this computation \textit{without} appealing to the
  fundamental theorem of algebra?
\end{problem}

\begin{problem}\label{uniformly-approximate-conj}For $\epsilon > 0$, is there a polynomial $p \in \mathbb{C}[z]$ so
  that $\abs{p(z) - \conj{z}} < \epsilon$ for $z \in B_1(0)$?  In
  other words, can we uniformly approximate $\conj{z}$ by a polynomial
  in $z$?
\end{problem}

\section{Prove or Disprove and Salvage if Possible}

\begin{problem}\label{entire-is-dense}If $f : \C \to \C$ is
  holomorphic, then the image of $f$ is dense in $\C$.
\end{problem}

\begin{problem}\label{identity-dominate-entire}Suppose $f : \C \to \C$
  is holomorphic and for all $z \in \C$ we have
  $\abs{f(z)} \leq \abs{z}$.  Then $f(z) = \lambda z$ for some
  $\lambda \in \C$.
\end{problem}

\begin{problem}
  There is a nonconstant holomorphic function $f : \C \to \C$ so that
  for all $n \in \Z$ we have $f(z + n) = f(z)$.  (Such a function is
  \textbf{periodic}.)
\end{problem}

\begin{problem}\label{doubly-periodic}There is a nonconstant holomorphic function $f : \C \to \C$ so that
  for all $\omega \in \Z[i]$ we have $f(z + \omega) = f(z)$.  (Such a
  function is \textbf{doubly periodic}.)
\end{problem}

\begin{problem}\label{maximum-modulus-principle}Consider a connected open subset $U \subset \C$ and a holomorphic
  function $f : U \to \C$.  If there is a point $z_0 \in U$ so that
  for all $z \in U$ we have $\abs{f(z_0)} \geq \abs{f(z)}$, then $f$
  is constant.  \textit{Hint:} show that $g(z) = \abs{f(z)}$ is
  constant and invoke \ref{open-mapping-theorem-preview}.  This is the
  \textbf{maximum modulus principle}.
\end{problem}


\end{document}

