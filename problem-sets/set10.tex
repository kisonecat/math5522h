\documentclass{homework}
\course{Math 5522H}
\author{Jim Fowler}
\usepackage{amsmath}
\DeclareMathOperator{\Mat}{Mat}
\DeclareMathOperator{\End}{End}
\DeclareMathOperator{\Hom}{Hom}
\DeclareMathOperator{\id}{id}
\DeclareMathOperator{\image}{im}
\DeclareMathOperator{\rank}{rank}
\DeclareMathOperator{\nullity}{nullity}
\DeclareMathOperator{\trace}{tr}
\DeclareMathOperator{\Spec}{Spec}
\DeclareMathOperator{\Sym}{Sym}
\DeclareMathOperator{\pf}{pf}
\DeclareMathOperator{\Ortho}{O}
\DeclareMathOperator{\diam}{diam}
\DeclareMathOperator{\Real}{Re}
\DeclareMathOperator{\Imag}{Im}
\DeclareMathOperator{\Arg}{Arg}

\newcommand{\C}{\mathbb{C}}
\newcommand{\R}{\mathbb{R}}
\newcommand{\Z}{\mathbb{Z}}
\newcommand{\N}{\mathbb{N}}


\DeclareMathOperator{\sla}{\mathfrak{sl}}
\newcommand{\norm}[1]{\left\lVert#1\right\rVert}
\newcommand{\transpose}{\intercal}

\newcommand{\conj}[1]{\bar{#1}}
\newcommand{\abs}[1]{\left|#1\right|}

%%% My commands, for solutions %%%

\DeclareMathOperator{\Log}{Log}
% To write df/(dx), use \pfrac{f}{x}
\newcommand{\pfrac}[2]{\frac{\partial #1}{\partial #2}}
% Partial derivative. To take d^2f/(dxdy), use \ppfrac[y]{f}{x}
% To take d^2f/(dx^2), use ppfrac{f}{x}
\newcommand{\ppfrac}[3][]{\frac{\partial^2 #2}{\ifthenelse{\isempty{#1}}{\partial #3^2}{\partial #3\partial #1}}}
 \newenvironment{solution}
  {\renewcommand\qedsymbol{$\blacksquare$}\begin{proof}[Solution]}
  {\end{proof}}






\DeclareMathOperator{\Res}{Res}

\begin{document}
\maketitle

\begin{inspiration} % https://frinkiac.com/caption/S01E04/240324
  As far as anybody knows, we're a nice normal family.
\byline{Homer Simpson in S01E04}
\end{inspiration}

\section{Terminology}

\begin{problem}
  What is a \textbf{normal family} of holomorphic functions on an open set $U$?
\end{problem}

\begin{problem}
  What does it mean to say that the infinite product
  \(
    \prod_{n=1}^\infty \left( 1 + a_n \right)
  \)
  converges?
\end{problem}

\section{Numericals}

\begin{problem}
  For which $z$ does the series
  \[
    \sum_{n=0}^\infty \frac{\cos \left( nz \right)}{n!}
  \]
  converge?

  Does it converge to a holomorphic function you recognize?
\end{problem}

\begin{problem}
  For which $z$ does the series
  \[
    \sum_{0 \neq n \in \Z} \left( \frac{1}{z+n} - \frac{1}{n} \right)
  \]
  converge?

  Does it converge to a meromorphic function you recognize?
  \textit{Hint:} \ref{residues-all-one}.
\end{problem}

\section{Exploration}

\begin{problem}\label{poisson-summation}Describe conditions on $f$ so
  that, for a suitable $\gamma_1$ and $\gamma_2$,
  \[
    \sum_{n=-\infty}^\infty f(n) = \int_{\gamma_1} \frac{f(z)}{e^{2\pi i z} - 1} \, dz - \int_{\gamma_2} \frac{f(z)}{e^{2\pi i z} - 1} \, dz.
  \]
  Then expand $1/(e^{2\pi i z} - 1)$ as a geometric series to deduce the \textbf{Poisson summation formula}
  \[
    \sum_{n=-\infty}^\infty f(n) = \sum_{n=-\infty}^\infty \hat{f}(n) 
  \]
  where $\hat{f}$ is the Fourier transform, i.e.,
  \[
    {\hat {f}}(\xi ) := \int _{-\infty }^{\infty} f(x) \, e^{-2\pi ix \xi} \,dx.
  \]
\end{problem}


\begin{problem}\label{modularity}For $a > 0$ define
  $\vartheta(a) = \sum_{n=-\infty}^\infty e^{-a \pi z^2}$.  Recalling
  \ref{fourier-transform-itself} shows that if $f(z) = e^{-a \pi z^2}$
  we can compute $\hat{f}(z)$.  Use $\hat{f}(z)$ and
  \ref{poisson-summation} to verify
  \[
    \vartheta(a) = \vartheta(1/a) / \sqrt{a}.
  \]
\end{problem}

\begin{problem}
  Having just celebrated $\pi$-day, some computer calculations revealed
\begin{align*}
  \vartheta(1/(4\pi)) = \sum_{n=-\infty}^\infty e^{-n^2/4} &=
3.544907701811032\textbf{10533931955126186}\ldots \\
  2\sqrt{\pi} &=
3.544907701811032\textbf{05459633496668229}\ldots
\end{align*}
Is my computer broken?  (For more, see \texttt{https://arxiv.org/abs/1809.10907}.)
\end{problem}

\begin{problem}
  Define
  \[
    \wp(z) = \frac{1}{z^2} + \sum_{0 \neq \lambda \in \Z[i]} \left( \frac{1}{(z - \lambda)^2} - \frac{1}{\lambda^2} \right)
  \]
  which is \textbf{Weierstrass' elliptic function} on the square lattice.  Use \ref{sum-one-over-gaussian-integers} to verify that $\wp(z)$ converges.
\end{problem}

\begin{problem}\label{elliptic-derivative-periodic}Compute $\wp'(z)$ by differentiating term-by-term to show that $\wp'(z + \lambda) = \wp'(z)$ for $\lambda \in \Z[i]$.
\end{problem}

\begin{problem}
  Compare $\wp(z)$ and $\wp(-z)$.  Use this to relate $\wp(1/2)$ and $\wp(-1/2)$ and to relate $\wp(i/2)$ and $\wp(-i/2)$ and with \ref{elliptic-derivative-periodic}, conclude that $\wp(z + \lambda) = \wp(z)$ for $\lambda \in \Z[i]$.
\end{problem}

\section{Prove or Disprove and Salvage if Possible}

\begin{problem}\label{differentiating-taylor-series}If
  $f(z) = \sum_{n=0}^\infty a_n z^n$ has radius of convergence $r$,
  then the series
  \[
    g(z) = \sum_{n=1}^\infty n a_n z^{n-1} 
  \]
  has radius of convergence $r$ and if $|z| < r$ then $f'(z) = g(z)$.
\end{problem}

\begin{problem}\label{sum-one-over-gaussian-integers}For $n > 2$, the
  series \( \displaystyle\sum_{0 \neq \lambda \in \Z[i]} \frac{1}{|\lambda|^n} \).
  converges.
\end{problem}

\begin{problem}\label{elliptic-more-than-two-poles}There is no meromorphic function $f : \C \to \hat{\C}$ having simple poles at $a+bi \in \Z[i]$ and satisfying $f(z) = f(z+a+bi)$ for all $z \in \C$ and $a+bi \in \Z[i]$.
\end{problem}

\begin{problem}\label{infinite-product-sine}For all $z \in \C$, we have 
  \(
    \sin \left( \pi z \right) = \pi \displaystyle\prod_{n=1}^\infty \left( 1 - \frac{z^2}{n^2} \right)
  \). % missing factor of z
\end{problem}

\begin{problem}\label{normal-family-example}Define
  $f_w : B_1(0) \to \C$ by $f_w(z) = z/(z-w)$.  The family of
  functions $\mathcal{F} := \{ f_w \mid w \in \C \}$ is a normal
  family.
\end{problem}

\begin{problem}\label{derivatives-normal-then-not-normal}Suppose $\mathcal{F}$ is a family of functions defined on the domain $B_1(0)$.  If $\mathcal{F}' := \{ f' \mid f \in \mathcal{F} \}$, the
  family consisting of derivatives of functions in the family
  $\mathcal{F}$, is normal, then the original family $\mathcal{F}$ is
  normal. % assume that \{ f(0) \mid f \in \mathcal{F} \} is bounded
\end{problem}

\end{document}
