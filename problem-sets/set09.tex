\documentclass{homework}
\course{Math 5522H}
\author{Jim Fowler}
\usepackage{amsmath}
\DeclareMathOperator{\Mat}{Mat}
\DeclareMathOperator{\End}{End}
\DeclareMathOperator{\Hom}{Hom}
\DeclareMathOperator{\id}{id}
\DeclareMathOperator{\image}{im}
\DeclareMathOperator{\rank}{rank}
\DeclareMathOperator{\nullity}{nullity}
\DeclareMathOperator{\trace}{tr}
\DeclareMathOperator{\Spec}{Spec}
\DeclareMathOperator{\Sym}{Sym}
\DeclareMathOperator{\pf}{pf}
\DeclareMathOperator{\Ortho}{O}
\DeclareMathOperator{\diam}{diam}
\DeclareMathOperator{\Real}{Re}
\DeclareMathOperator{\Imag}{Im}
\DeclareMathOperator{\Arg}{Arg}

\newcommand{\C}{\mathbb{C}}
\newcommand{\R}{\mathbb{R}}
\newcommand{\Z}{\mathbb{Z}}
\newcommand{\N}{\mathbb{N}}


\DeclareMathOperator{\sla}{\mathfrak{sl}}
\newcommand{\norm}[1]{\left\lVert#1\right\rVert}
\newcommand{\transpose}{\intercal}

\newcommand{\conj}[1]{\bar{#1}}
\newcommand{\abs}[1]{\left|#1\right|}

%%% My commands, for solutions %%%

\DeclareMathOperator{\Log}{Log}
% To write df/(dx), use \pfrac{f}{x}
\newcommand{\pfrac}[2]{\frac{\partial #1}{\partial #2}}
% Partial derivative. To take d^2f/(dxdy), use \ppfrac[y]{f}{x}
% To take d^2f/(dx^2), use ppfrac{f}{x}
\newcommand{\ppfrac}[3][]{\frac{\partial^2 #2}{\ifthenelse{\isempty{#1}}{\partial #3^2}{\partial #3\partial #1}}}
 \newenvironment{solution}
  {\renewcommand\qedsymbol{$\blacksquare$}\begin{proof}[Solution]}
  {\end{proof}}






\DeclareMathOperator{\Res}{Res}

\begin{document}
\maketitle

\begin{inspiration}
  Luck is the residue of design
  \byline{Branch Rickey}
\end{inspiration}

\section{Terminology}

\begin{problem}
  Define the \textbf{residue} of $f$ at the point $z$, which we write $\Res(f,z)$.
\end{problem}

\section{Numericals}

\begin{problem}
  Evaluate $\Res(f,z)$ for the function $f(z) = \displaystyle\frac{e^z}{z^2-1}$.
\end{problem}

\begin{problem}\label{residues-all-one}Evaluate $\Res(f,z)$ for the
  function $f(z) = \pi \cot (\pi z)$.
\end{problem}

\begin{problem}\label{residue-coth}Compute $\Res(f,\pm bi)$ for the
  function
  \[
    f(z) = \frac{\pi \cot(\pi z)}{z^2 + b^2}.
  \]
\end{problem}

\begin{problem}
  Evaluate the integrals
  \[
    \int_{-\infty}^\infty \frac{\sin x}{1+x^2} \, dx \mbox{ and }
    \int_{-\infty}^\infty \frac{\cos x}{1+x^2} \, dx.
  \]
\end{problem}

\begin{problem}\label{integral-for-euler-reflection}For a real number
  $\lambda \in (0,1)$, evaluate
  \[
    \int_{-\infty}^\infty \frac{e^{\lambda x}}{1 + e^x} \, dx.
  \]
\end{problem}

\begin{problem}
  Evaluate the integral $\displaystyle\int_{0}^{2\pi} \frac{1}{3 + \sin^2 x} \, dx$.
\end{problem}

\begin{problem}
  Evaluate the integral $\displaystyle \int_0^\pi \log \sin x \, dx$.
\end{problem}

\begin{problem}
  Evaluate the integral $\displaystyle\int_{-\infty}^{\infty} \frac{1-x}{1-x^7} \, dx$.
\end{problem}

\begin{problem}
  Show that $f(z) = z^5 + 15z - 1$ has five zeros in $B_2(0)$, and one zero in $B_{1/15}(0)$.
\end{problem}

\begin{problem}\label{form-at-infinity}Set $w = 1/z$ and compute $f(w) \, dw$ in terms of $f(z) \, dz$.
\end{problem}

\section{Exploration}

\begin{problem}
  Fix $w \in \R$ with $w>0$.  By the intermediate value theorem, the
  polynomial $f(z) = z^4 + 4w^3 z - 1$ has a real root in the interval
  $(-\infty,0)$ and another in $(0,1)$, along with two complex roots
  $a \pm bi$.  Use Gauss-Lucas (\ref{gauss-lucas}) and Rouch\'e's
  (\ref{rouches-theorem}) theorem to describe a subset of $\C$
  containing $a\pm bi$.
\end{problem}

\begin{problem}
  What is the correct definition of $\Res(f,\infty)$?  Generally, we
  would shift the viewport by replacing $f$ with the function
  $g(z) = f(1/z)$ and then we study $g$ near $z = 0$ in order to
  investigate ``$f$ near $\infty$.''  Does \ref{form-at-infinity} help
  here?
\end{problem}

\begin{problem}\label{rouches-theorem}Suppose $U$ is an open set containing the closed disk $D_r(z_0)$ and
  $f, g : U \to \C$ are holomorphic functions satisfying
  \[
    \abs{f(z)} > \abs{g(z)}
  \]
  for $z \in \partial D_r(z_0)$.  Prove \textbf{Rouch\'e's theorem}
  that $f$ and $f+g$ have the same number of zeros, counted with
  multiplicity, in $B_r(z_0)$.  (This is sometimes called the dog
  leash theorem---can you see why?)
\end{problem}

\begin{problem}
  Use \ref{rouches-theorem} to give another proof of the Fundamental Theorem of Algebra.
\end{problem}

\begin{problem}
  It is important to recognize patterns in how residues equip us to evaluate integrals, so that, when presented with a fresh integration problem, we have some ideas about the best tool for the task.  Develop a tool for evaluating
  \[
    \int_0^{2\pi} f(\cos \theta,\sin \theta) \, d\theta
  \]
  via residues.  Here $f$ is a rational function, explain how to make
  the substitution $z = e^{i\theta}$ to rewrite
  $f(\cos \theta,\sin \theta)$ in terms of $z$, and
  similarly rewrite $d\theta$ in terms of
  $dz$ to reduce the given integral to a certain contour integral.
\end{problem}

\begin{problem}\label{summation-theorem}Sometimes the residue calculus
  permits us to sum series.

  Suppose $f$ is meromorphic with poles $z_1,\ldots,z_N$ with $z_j \not\in \Z$.  Our goal is a formula
  \[
    \sum_{n=-\infty}^{\infty} f(n) = - \sum_{j=1}^N \Res(g, z_j)
  \]
  where $g(z) = \pi \cot(\pi z) f(z)$.  \textit{Hint:} Consider a square contour $R$ centered at the origin with side-length $2N+1$.  You may assume that $\pi \cot(\pi z)$ is bounded on $R$ independent of $N$.  You will then need to impose some bound on $f$ on the contour $R$.
\end{problem}

\begin{problem}
  Apply \ref{residue-coth} and \ref{summation-theorem} to evaluate
  \[
    \sum_{n=-\infty}^\infty \frac{1}{n^2 + b^2}.
  \]
\end{problem}

\section{Prove or Disprove and Salvage if Possible}

\begin{problem}\label{zero-residue-not-enough}Suppose $f : D_r(z_0) \to \C$ is holomorphic.  The residue
  $\Res(f,z_0)$ vanishes if and only if the singularity $z_0$ is
  removable.
\end{problem}

\begin{problem}\label{open-mapping-theorem}If $U \subset \C$ is open
  and $f : U \to \C$ is holomorphic, then $f(U)$ is
  open. % missing nonconstant, can be proved from rouche
\end{problem}

\end{document}
