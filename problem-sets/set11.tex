\documentclass{homework}
\course{Math 5522H}
\author{Jim Fowler}
\usepackage{amsmath}
\DeclareMathOperator{\Mat}{Mat}
\DeclareMathOperator{\End}{End}
\DeclareMathOperator{\Hom}{Hom}
\DeclareMathOperator{\id}{id}
\DeclareMathOperator{\image}{im}
\DeclareMathOperator{\rank}{rank}
\DeclareMathOperator{\nullity}{nullity}
\DeclareMathOperator{\trace}{tr}
\DeclareMathOperator{\Spec}{Spec}
\DeclareMathOperator{\Sym}{Sym}
\DeclareMathOperator{\pf}{pf}
\DeclareMathOperator{\Ortho}{O}
\DeclareMathOperator{\diam}{diam}
\DeclareMathOperator{\Real}{Re}
\DeclareMathOperator{\Imag}{Im}
\DeclareMathOperator{\Arg}{Arg}

\newcommand{\C}{\mathbb{C}}
\newcommand{\R}{\mathbb{R}}
\newcommand{\Z}{\mathbb{Z}}
\newcommand{\N}{\mathbb{N}}


\DeclareMathOperator{\sla}{\mathfrak{sl}}
\newcommand{\norm}[1]{\left\lVert#1\right\rVert}
\newcommand{\transpose}{\intercal}

\newcommand{\conj}[1]{\bar{#1}}
\newcommand{\abs}[1]{\left|#1\right|}

%%% My commands, for solutions %%%

\DeclareMathOperator{\Log}{Log}
% To write df/(dx), use \pfrac{f}{x}
\newcommand{\pfrac}[2]{\frac{\partial #1}{\partial #2}}
% Partial derivative. To take d^2f/(dxdy), use \ppfrac[y]{f}{x}
% To take d^2f/(dx^2), use ppfrac{f}{x}
\newcommand{\ppfrac}[3][]{\frac{\partial^2 #2}{\ifthenelse{\isempty{#1}}{\partial #3^2}{\partial #3\partial #1}}}
 \newenvironment{solution}
  {\renewcommand\qedsymbol{$\blacksquare$}\begin{proof}[Solution]}
  {\end{proof}}






\DeclareMathOperator{\Res}{Res}

\usepackage[symbol]{footmisc}
\renewcommand{\thefootnote}{\fnsymbol{footnote}}

\begin{document}
\maketitle

\begin{inspiration}
  Wenn ich nur erst die S\"atze habe! Die Beweise werde ich schon
  finden.\footnote{``If only I first had the theorems, then I'd find
    the proofs somehow.''  Riemann knew all too well the challenge of
    PODASIPs.} \byline{Bernhard Riemann}
\end{inspiration}

\section{Terminology}

\begin{problem}
  When $\Real s > 0$, what is the Gamma function $\Gamma(s)$?
\end{problem}

\begin{problem}
  When $\Real s > 1$, what is the Riemann zeta function $\zeta(s)$?
\end{problem}

\begin{problem}
  What is the von Mangoldt function $\Lambda(n)$?
\end{problem}

\section{Numericals}

\begin{problem}
  Compute $\Res(\Gamma,s)$ for $s \in \{ 0, -1, -2, \ldots \}$.
\end{problem}

\begin{problem}Evaluate $\Res(f,s)$ for the
  function $f(s) = \Gamma(s) \Gamma(1-s)$.
\end{problem}

\begin{problem} % be careful to get the sign right!
  Use \ref{euler-reflection-formula} to compute $\Gamma(1/2)$.  
\end{problem}

\section{Exploration}

\begin{problem}
  In previous courses, you have seen that
  $\Gamma(s+1) = s \cdot \Gamma(s)$.  Use this fact to explain how to
  define a meromorphic function on $\C$ agreeing with $\Gamma(s)$ for
  $\Real s > 0$.  This is \textbf{analytic continuation} which succeeds
  here by using a \textbf{functional equation}.
\end{problem}

\begin{problem}\label{start-of-reflection-formula}For $\lambda \in (0,1)$, evaluate
  \[
    \int_{0}^\infty \frac{u^{\lambda - 1}}{1 + u} \, du.
  \]
  \textit{Hint:} make the substitution $u = e^x$ and invoke \ref{integral-for-euler-reflection}.
\end{problem}

\begin{problem}\label{integrate-gamma-one-minus-s}For $s \in (0,1)$ and positive $t \in \R$, show that
  \[
    \Gamma(1 - s) = t \int_0^\infty (xt)^{-s} e^{-xt}\, dx.
  \]
\end{problem}

\begin{problem}\label{euler-reflection-formula}Combine \ref{start-of-reflection-formula} and \ref{integrate-gamma-one-minus-s} to conclude that
  \[
    \Gamma(1-s) \, \Gamma(s) = \frac{\pi}{\sin \left( \pi s \right)}.
  \]
  This is \textbf{Euler's reflection formula}.
\end{problem}

\begin{problem}
  Show that the series
  \[\displaystyle\sum_{n=1}^\infty \displaystyle\frac {1}{n^{s}}\]
  converges to a holomorphic function when $\Real s > 1$.
\end{problem}

\begin{problem}\label{euler-product-formula}Recall that you defined
  \(\displaystyle\prod_{n=1}^\infty \left( 1 + a_n \right)\) in
  \ref{terminology-infinite-product}.  Show that when $\Real s > 1$, we
  have
  \[
    \frac{1}{\zeta(s)} = \displaystyle\prod_{n=1}^\infty \left( 1 - {p_n}^s \right),
  \]
  where $p_n$ is the $n$th prime number.
\end{problem}

\begin{problem}
  Use \ref{euler-product-formula} and \ref{nonzero-infinite-product}
  to show that if $\Real s > 1$ then $\zeta(s) \neq 0$.
\end{problem}

\begin{problem}
  Use \ref{euler-product-formula} to show that
  \[
    \Real \left( \frac{ \zeta'(a+bi) }{ \zeta(a+bi) } \right) = - \sum_{n=1}^\infty \frac{\Lambda(n) \, \cos \left( b \log n \right)}{n^{a}}.
  \]
\end{problem}

\begin{problem}
  Use the trigonometric fact $3 + 4 \cos \theta + \cos (2\theta) \geq 0$ to show that
  \[
    \Real \left( 3 \frac{ \zeta'(a) }{ \zeta(a) } + 4 \frac{ \zeta'(a+bi) }{ \zeta(a+bi) } +  \frac{ \zeta'(a+2bi) }{ \zeta(a+2bi) } \right) \leq 0.
  \]
  Knowing $\zeta$ has a simple pole at $s = 1$ and assuming
  $\zeta(1+bi) = 0$ for $b \in \R$, define 
  \[
    f(a) = \zeta(a)^3 \cdot \zeta(a+bi)^4 \cdot \zeta(a + 2bi)
  \]
  and study $f$ near $a = 1$ to uncover a contradiction.  You will
  have shown that $\zeta(s) \neq 0$ when $\Real s = 1$.
\end{problem}

\section{Prove or Disprove and Salvage if Possible}

\begin{problem}\label{nonzero-infinite-product}For a sequence $a_n \neq 0$ of complex numbers, suppose
  $\displaystyle\sum_{n=1}^\infty |a_n|$ converges.  Then
  $\displaystyle\prod_{n=1}^\infty \left( 1 + a_n \right)$ converges
  to a nonzero quantity.
\end{problem}

\begin{problem} % missing pole at s = 1
  Define the \textbf{Dirichlet eta function} via
  \[
    \eta(s) = := \sum_{n=1}^\infty \frac{(-1)^n}{n^s}.
  \]
  This series converges to an analytic function when $\Real s > 0$.
\end{problem}

\begin{problem}
  The Riemann zeta and Dirichlet eta functions are related via
  \[
    \left( 1 - 2^{1-s} \right) \zeta(s) = \eta(s).
  \]
\end{problem}

\end{document}

%vc
%proof of picard's theorem



%prove liouville via singularities for analytic functions: f(1/z)  extends holomorphically around zero, it is easy to observe that f achieves its supremum, either on C, or ``at infinity'' (meaning that the extension of f(1/z) achieves a local supremum at 0). \in either case, f has to be constant



%proof of fundamental theorem of algebra via max modulus

%https://en.wikipedia.org/wiki/Abel%E2%80%93Plana_formula

% Stieltjes inversion formula t

% can cts functions on the disk be uniformly approximated by an element $\C[z]$?

% suppose $|f(z) - 1| < 1$ \in a region $\Omega$, show int f' / dz = 0
% suppose $|f(z) - 1| < f(z) + 1$ \in a region $\Omega$, show int f' / dz = 0

% https://en.wikipedia.org/wiki/Jordan%27s_lemma

% fourier series, for holomorphic functions \in the disk, the coefficientgs are positive
% parseval's theorem

% injective entire functions are affine; apply casorati-weierstrass to f(1/z)
