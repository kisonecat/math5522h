\documentclass{homework}
\course{Math 5522H}
\author{Jim Fowler}
\usepackage{amsmath}
\DeclareMathOperator{\Mat}{Mat}
\DeclareMathOperator{\End}{End}
\DeclareMathOperator{\Hom}{Hom}
\DeclareMathOperator{\id}{id}
\DeclareMathOperator{\image}{im}
\DeclareMathOperator{\rank}{rank}
\DeclareMathOperator{\nullity}{nullity}
\DeclareMathOperator{\trace}{tr}
\DeclareMathOperator{\Spec}{Spec}
\DeclareMathOperator{\Sym}{Sym}
\DeclareMathOperator{\pf}{pf}
\DeclareMathOperator{\Ortho}{O}
\DeclareMathOperator{\diam}{diam}
\DeclareMathOperator{\Real}{Re}
\DeclareMathOperator{\Imag}{Im}
\DeclareMathOperator{\Arg}{Arg}

\newcommand{\C}{\mathbb{C}}
\newcommand{\R}{\mathbb{R}}
\newcommand{\Z}{\mathbb{Z}}
\newcommand{\N}{\mathbb{N}}


\DeclareMathOperator{\sla}{\mathfrak{sl}}
\newcommand{\norm}[1]{\left\lVert#1\right\rVert}
\newcommand{\transpose}{\intercal}

\newcommand{\conj}[1]{\bar{#1}}
\newcommand{\abs}[1]{\left|#1\right|}

%%% My commands, for solutions %%%

\DeclareMathOperator{\Log}{Log}
% To write df/(dx), use \pfrac{f}{x}
\newcommand{\pfrac}[2]{\frac{\partial #1}{\partial #2}}
% Partial derivative. To take d^2f/(dxdy), use \ppfrac[y]{f}{x}
% To take d^2f/(dx^2), use ppfrac{f}{x}
\newcommand{\ppfrac}[3][]{\frac{\partial^2 #2}{\ifthenelse{\isempty{#1}}{\partial #3^2}{\partial #3\partial #1}}}
 \newenvironment{solution}
  {\renewcommand\qedsymbol{$\blacksquare$}\begin{proof}[Solution]}
  {\end{proof}}







\begin{document}
\maketitle

\begin{inspiration}
It is singularity which often makes the worst part of our suffering.\\
\byline{Jane Austen}, not talking about this problem set.
\end{inspiration}

\section{Terminology}

This week, there is a lot of new terminology to classify the various
sorts of singularities we might encounter.  For full credit, be sure
to give careful precise definitions.

\begin{problem}
  What is an \textbf{isolated singularity}?
\end{problem}

\begin{problem}
  What is meant by the \textbf{order} (or \textbf{multiplicity}) of a zero?  Of a pole?
\end{problem}

\begin{problem}
  What is a \textbf{removable singularity}?
\end{problem}

\begin{problem}
  What does it mean to say that a function $f : \C \to \C$ has a zero of order $n$ at infinity?  Has a pole of order $n$ at infinity?  
\end{problem}

\begin{problem}
What is a \textbf{meromorphic function}?
\end{problem}

\section{Numericals}

\begin{problem}
  Let $f(z) = e^{z \sin^2 z} - 1$.  What is the order of the zero at $z = 0$?
\end{problem}

\begin{problem}
  Let $f(z) = \left( \cos z \right) - 1 - z^2/2$, and compute
  \[
    \int_\gamma \frac{f'(z)}{f(z)} \, dz
  \]
  for $gamma : [0,2\pi]$ given by $\gamma(\theta) = e^{i\theta}$.
\end{problem}

\section{Exploration}

\begin{problem}
  Explain the topology on $\mathbb{C} \cup \{ \infty \}$, the \textbf{Riemann sphere}.
\end{problem}

\begin{problem}\label{jordans-lemma}
  Prove \textbf{Jordan's lemma}; next week, this lemma will help us estimate integrals over the contour $\gamma : [0,\pi] \to \C$ given by $\gamma(\theta) = re^{i\theta}$.
  Specifically, show that if $a > 0$, then
  \[
    \abs{\int_\gamma e^{iaz} \, g(z) \, dz } \leq \frac{\pi}{a} \sup_{\theta \in [0,\pi]} \abs{g(\gamma(\theta))}.
    \]
\end{problem}

\begin{problem}\label{riemann-removable-singularity}
  For an open set $U \ni z_0$, suppose
  $f : U \setminus \{z_0\} \to \C$ is holomorphic.  Show that
  $\lim_{z\to z_0} (z-z_0)f(z)=0$ if and only $f$ extends to a
  holomorphic function $F : U \to \C$.
\end{problem}

\begin{problem}\label{idempotent-entire}Describe holomorphic functions
  $f : \C \to \C$ with the property that $f(f(z)) = f(z)$ for all
  $z \in \C$.
\end{problem}

\begin{problem}
  Suppose $U$ is a disk and $f : U \to \C$ is holomorphic with
  finitely many zeros, and repeatedly invoke \ref{factor-theorem} to
  explain why you can find $z_1,z_2,\ldots,z_n \in \C$ and write
  \[
    f(z) = (z-z_1)(z-z_2) \cdots (z-z_n) \, g(z)
  \]
  for a nowhere-vanishing analytic function $g : U \to \C$.
\end{problem}

\begin{problem}\label{argument-principle-zeros}
  Continuing as above, compute $f'(z)/f(z)$ in terms of $z_1,z_2,\ldots,z_n \in \C$ and $g'(z)/g(z)$, and evaluate
  \[
    \frac{1}{2\pi i} \int_\gamma \frac{f'(z)}{f(z)} \, dz
  \]
  in terms of the winding numbers $n(\gamma,z_j)$.
\end{problem}

\begin{problem}
  Let's justify the terminology that a ``zero of multiplicity $n$''
  really means there are $n$ solutions to a certain equation.  Suppose
  the holomorphic function $f : B_1(0) \to \C$ has a zero of order $n$
  at zero, i.e., suppose $f(z) = z^n g(z)$ for a holomorphic
  $g : B_1(0) \to \C$ with $g(0) \neq 0$.  For all sufficiently small
  $\epsilon > 0$, find $\delta > 0$ so that for all
  $w \in B_\epsilon(0) - \{0\}$, the set
  $f^{-1}(\{w\}) \cap B_\delta(0)$ consists of $n$ elements.
\end{problem}

\section{Prove or Disprove and Salvage if Possible}

\begin{problem}\label{factor-theorem}
  Suppose $U \subset \C$ is open, and $f : U \to \C$ is analytic, and for some $z_0 \in U$, we have $f(z_0) = 0$.  Then there is a positive $m \in \Z$ so that
  \[
    g(z) := \frac{f(z)}{(z-z_0)^m}
  \]
  yields an analytic function $g : U \to \C$ which does not vanish at $z_0$.
\end{problem}

\begin{problem}\label{entire-dominate-entire}Suppose $f, g : \C \to \C$ are holomorphic and for all $z \in \C$ we have $\abs{f(z)} \leq \abs{g(z)}$.  In this case, we say that $g$ dominates $f$.  Then $f(z) = \lambda \cdot g(z)$ for some $\lambda \in \C$.  (Compare \ref{identity-dominate-entire}.)
\end{problem}

\begin{problem}
  If $f : \C \to \C$ has a pole of order $n$ at infinity, then $f$ is a polynomial of degree at most $n$.
\end{problem}

\begin{problem}\label{casorati-weierstrass}
  Suppose $f : U \setminus \{ z_0 \} \to \C$ is holomorphic with an essential singularity at $z_0 \in U$.  If $V \subset U$ is a neighborhood of $z_0$, then $f(V \setminus \{ z_0 \})$ is dense in $\C$.
\end{problem}

\begin{problem}
  There exists a holomorphic function $f : \C \to \C$ so that both $f$
  and $z \mapsto e^{f(z)}$ have poles at zero.
\end{problem}

\begin{problem}
  There exists a nowhere-vanishing holomorphic function
  $f : \C \to \C$ such that $\lim_{z \to \infty} f(z) = \infty$.
\end{problem}

\end{document}
