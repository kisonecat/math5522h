\documentclass{homework}
\course{Math 5522H}
\author{Jim Fowler}
\input{preamble}
\DeclareMathOperator{\Res}{Res}

\usepackage[symbol]{footmisc}
\renewcommand{\thefootnote}{\fnsymbol{footnote}}

\begin{document}
\maketitle

% https://www.maths.tcd.ie/pub/HistMath/People/Riemann/Grund/Grund.pdf
% page 37
\begin{inspiration}
  Zwei gegebene einfach zusammenh\"angende ebene Fl\"achen k\"onnen
  stets so auf einander bezogen werden, dass jedem Punkte der einen
  Ein mit ihm stetig fortr\"uckender Punkt der anderen entspricht und
  ihre entsprechenden kleinsten Theile \"ahnlich sind; und zwar kann
  zu Einem innern Punkte und zu Einem Begrenzungspunkte der
  entsprechende beliebig gegeben werden; dadurch aber ist f\"ur all
  Punkte die Beziehung bestimmt.\footnote{``Two given simply-connected
    plane Surfaces can always be related to each other, so that each
    Point in the One with a continuously advancing Point corresponds
    to the other and their corresponding smallest Parts are similar
    [meaning the map is conformal?]; and namely, the correspondence
    between Inner Points and Limit Points can be given arbitrarily;
    but because of that the Relationship for all Points is
    determined.''}  \byline{Bernhard Riemann, in his 1851 PhD thesis
    \textit{Grundlagen f\"ur eine allgemeine Theorie der Functionen
      einer ver\"anderlichen complexen Gr\"osse.}}
\end{inspiration}

\section{Terminology}

\begin{problem}
  What is a \textbf{conformal map}?  Recall
  \ref{holomorphic-is-conformal} and note that various people define
  ``conformal map'' differently.
\end{problem}

\begin{problem}
  What is a \textbf{conformal equivalence} between two open subsets of $\C$?
\end{problem}

\begin{problem}
  How does the cross-ratio extend to $\hat{\C} = \C \cup \{\infty\}$?
  Recall \ref{cross-ratio}.
\end{problem}

\section{Numericals}

Feel free to express your
biholomorphisms as the composition of other maps.

\begin{problem}
  Consider the sector \(S_\theta := \{ z \in \C : 0 < |z| < 1 \mbox{ and } 0 < \Arg z < \theta \}\). Describe a biholomorphic map between $S_\theta$ and $S_\pi$.
\end{problem}

\begin{problem}
  Consider the half-disk
  \[
    D := \{ x + iy \in \C : x, y \in \R \mbox{ and } x^2+y^2 < 1 \mbox{ and } x > 0 \}
  \]
  and the first quadrant
  \[
    Q := \{ x + iy \in \C : x, y \in \R \mbox{ and } x > 0 \mbox{ and } y > 0 \}.
  \]
  Describe a biholomorphic map between $D$ and $Q$.
\end{problem}

\begin{problem}
  Again consider the half-disk $D$. Describe a biholomorphic map between $D$ and $B_1(0)$.
\end{problem}

\begin{problem}
  Consider the infinite horizontal strip
  \[
    S := \{ x + iy \in \C : x, y \in \R \mbox{ and } |y| < 1 \}.
  \]
  Describe a biholomorphic map between $S$ and $B_1(0)$.  
\end{problem}

\section{Exploration}

\begin{problem}
  There are 24 permutations of four distinct objects, and sometimes
  permuting the four inputs to the cross-ratio results in a different
  output.  Using the same four inputs, how many different outputs of
  the cross-ratio are possible?  How are these outputs related to each
  other?  There is a group structure here, giving rise to the
  \textbf{anharmonic group}.
\end{problem}

\begin{problem}\label{entire-injective-is-affine}A nonconstant entire function
  $f : \C \to \C$ is \textit{affine} if there are constants
  $a \in \C \setminus \{0\}$ and $b \in \C$ so that $f(z) = az + b$.
  
  Show that injective entire functions are affine.  \textit{Hint:}
  apply Casorati-Weierstrass to $z \mapsto f(1/z)$.
\end{problem}

\begin{problem}
  Using \ref{entire-injective-is-affine} and
  \ref{automorphisms-of-disk} and some dimension counting, conclude
  there is no biholomorphism between the unit ball $B_1(0)$ and the
  plane $\C$.  (Of course, this is \textit{much easier} to see via
  Liouville!)
\end{problem}

\begin{problem}
  Define \(B_{r,R}(0) := \{ z \in \C : r < |z| < R \}\). 
  Suppose $0 < r < R$.  Is there a biholomorphism between the
  punctured disk $B_{0,R}(0)$ and the annulus $B_{r,R}(0)$?

  You might say that the annulus is ``doubly-connected'' so we are
  exploring the extent to which something like the Riemann mapping
  theorem holds for regions a bit more complicated than
  simply-connected domains.
\end{problem}

\begin{problem}
  Is there a biholomorphism between the annulus $B_{r,R}(0)$ and the
  annulus $B_{\lambda r, \lambda R}(0)$?
\end{problem}

\begin{problem}
  Is there a biholomorphism between the annulus $B_{r,R}(0)$ and the
  annulus $B_{1/R,1/r}(0)$?
\end{problem}

\begin{problem}\label{annulus-to-punctured-disk}Suppose $r > 1$ and $R > 1$ and that there is a biholomorphism
  $f : B_{1,r}(0) \to B_{1,R}(0)$ sending the inside boundary to the
  inside boundary, i.e., for a sequence
  $z_1, z_2, \ldots \in B_{1,r}(0)$ with $\lim_n |z_n| = 1$, we have
  $\lim_n |f(z_n)| = 1$.
  
  Repeatdly reflect across the boundary to produce a biholomorphism
  $F : \C \setminus \{0\} \to \C \setminus \{0\}$ extending $f$.

  \textit{A technical point:} the Schwarz reflection principle, as we
  have stated it, requires $f$ to extend continuously to the boundary.
  This issue can be overcome here.
\end{problem}

\begin{problem}
  The function $F$ constructed in \ref{annulus-to-punctured-disk}
  extends to an entire function, which by
  \ref{entire-injective-is-affine} is affine.  Deduce
  \textbf{Schottky's theorem for annuli} describing exactly when
  annuli $B_{r,R}(0)$ and $B_{r',R'}(0)$ and are conformally
  equivalent.
\end{problem}

\section{Prove or Disprove and Salvage if Possible}

\begin{problem}
  A map $f : \C \to \C$ is conformal if and only if it is holomorphic.
\end{problem}

\begin{problem}
  For triples $z_1, z_2, z_3 \in \C$ and $w_1,w_2,w_3 \in \C$, there
  is a conformal map $f : \C \to \C$ with $f(z_1) = w_1$ and
  $f(z_2) = w_2$ and $f(z_3) = w_3$. % many issues!
\end{problem}

\begin{problem}
  If there is a biholomorphism $f : U \to V$ and $U$ is
  simply-connected, then $V$ is simply-connected.
\end{problem}


\end{document}


