\documentclass{homework}
\course{Math 5522H}
\author{Jim Fowler}
\usepackage{amsmath}
\DeclareMathOperator{\Mat}{Mat}
\DeclareMathOperator{\End}{End}
\DeclareMathOperator{\Hom}{Hom}
\DeclareMathOperator{\id}{id}
\DeclareMathOperator{\image}{im}
\DeclareMathOperator{\rank}{rank}
\DeclareMathOperator{\nullity}{nullity}
\DeclareMathOperator{\trace}{tr}
\DeclareMathOperator{\Spec}{Spec}
\DeclareMathOperator{\Sym}{Sym}
\DeclareMathOperator{\pf}{pf}
\DeclareMathOperator{\Ortho}{O}
\DeclareMathOperator{\diam}{diam}
\DeclareMathOperator{\Real}{Re}
\DeclareMathOperator{\Imag}{Im}
\DeclareMathOperator{\Arg}{Arg}

\newcommand{\C}{\mathbb{C}}
\newcommand{\R}{\mathbb{R}}
\newcommand{\Z}{\mathbb{Z}}
\newcommand{\N}{\mathbb{N}}


\DeclareMathOperator{\sla}{\mathfrak{sl}}
\newcommand{\norm}[1]{\left\lVert#1\right\rVert}
\newcommand{\transpose}{\intercal}

\newcommand{\conj}[1]{\bar{#1}}
\newcommand{\abs}[1]{\left|#1\right|}

%%% My commands, for solutions %%%

\DeclareMathOperator{\Log}{Log}
% To write df/(dx), use \pfrac{f}{x}
\newcommand{\pfrac}[2]{\frac{\partial #1}{\partial #2}}
% Partial derivative. To take d^2f/(dxdy), use \ppfrac[y]{f}{x}
% To take d^2f/(dx^2), use ppfrac{f}{x}
\newcommand{\ppfrac}[3][]{\frac{\partial^2 #2}{\ifthenelse{\isempty{#1}}{\partial #3^2}{\partial #3\partial #1}}}
 \newenvironment{solution}
  {\renewcommand\qedsymbol{$\blacksquare$}\begin{proof}[Solution]}
  {\end{proof}}







\DeclareMathOperator{\polylog}{Li}
\newcommand{\dilog}{\polylog_2}

\begin{document}
\maketitle

\begin{inspiration}
    Music is nothing but ratios and harmonic math, anyways.
\byline{Andrew Sega}
\end{inspiration}

With this being a ``short'' week because of the instructional break,
this problem set is concomitantly shorter.

\section{Terminology}

\begin{problem}
  What is a harmonic function?
\end{problem}

\begin{problem}
  Define the term \textbf{harmonic conjugate}.  (Recall \ref{harmonic-conjugate}.)
\end{problem}

\section{Numericals}

\begin{problem}\label{relate-fourier-and-taylor-series}Let $S^1 := \{(x,y) \in \R^2 : x^2 + y^2 = 1 \}$ and $D_1(0) := \{ (x,y) \in \R^2 : x^2 + y^2 \leq 1 \}$.  Find a continuous function $F : D_1(0) \to \R$ that is harmonic on the interior of $D_1(0)$ and that extends
  $f : S^1 \to \R$ given by
 \[
   f(\cos \theta,\sin \theta) = \sum_{k=0}^N \left(c_k \, \cos \left( k\theta \right) + s_k \, \sin \left( k \theta \right) \right).
 \]
 (This problem encourages you to relate Fourier series and Taylor series.)
\end{problem}

\section{Exploration}

\begin{problem}
  Recall \ref{where-converge-one-over-n}.  Consider the power series for the \textbf{dilogarithm} function,
  \[
    \dilog(z) = \sum_{n=1}^\infty \frac{z^n}{n^2}.
  \]
  What is its radius $r$ of convergence?  Where on the boundary of
  $B_r(0)$ does this series converge?  (Note that there is no pole on
  the boundary; not every singularity is a pole!)
\end{problem}

\begin{problem}\label{laplacian-via-wirtinger}Consider an open set
  $U, V \subset \C$ and a holomorphic function $f : U \to \C$.  Write
  the \textbf{Laplacian}
  \[
    \Delta := \frac{\partial^2}{\partial x^2} + \frac{\partial^2}{\partial y^2} 
  \]
  in terms of the mixed Wirtinger derivative
  $\displaystyle\frac{\partial^2}{\partial z \partial \conj{z}}$.
\end{problem}

\begin{problem}\label{composition-holomorphic-harmonic}Consider open
  sets $U, V \subset \C$ and a holomorphic function $f : U \to V$ and
  a harmonic function $g : V \to \R$.  Is the composition $g \circ f$
  a harmonic function on $V$?
\end{problem}

\section{Prove or Disprove and Salvage if Possible}

\begin{problem} % wrong sign
  If $f : U \to \R$ is a harmonic function on an open set $U \subset \R^2$, then 
  \[
    F(x+iy) := f_x(x,y) + i f_y(x,y) 
  \]
  defines a holomorphic function $F : U \to \C$, regarding $U$ as an
  open subset of $\C$.
\end{problem} 

\begin{problem}\label{maximum-principle}For an open set
  $U \subset \R^2$ and a harmonic function $f : U \to \R$, the
  function $f$ does not achieve a maximum. % nonconstant missing
\end{problem}

\begin{problem}\label{universal-taylor-series}Suppose
  $\displaystyle\sum_{n=0}^\infty a_n z^n$ is a power series with
  radius of convergence 1.  We say that series is \textbf{universal}
  if, for every $\epsilon > 0$ and $\delta > 0$, for every closed disk
  $D_r(a)$ with $\abs{a-1} > r$, for every holomorphic function
  $f : B_{r + \delta}(a) \to \C$, there exists $N$ so that
  \[
    \sup_{z \in D_r(a)} \abs{ f(z) - \sum_{n=0}^N a_n z^n } < \epsilon.
  \]

  There is a universal power series.
\end{problem}

\end{document}

