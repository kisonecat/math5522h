\documentclass{homework}
\usepackage{amsmath}
\DeclareMathOperator{\Mat}{Mat}
\DeclareMathOperator{\End}{End}
\DeclareMathOperator{\Hom}{Hom}
\DeclareMathOperator{\id}{id}
\DeclareMathOperator{\image}{im}
\DeclareMathOperator{\rank}{rank}
\DeclareMathOperator{\nullity}{nullity}
\DeclareMathOperator{\trace}{tr}
\DeclareMathOperator{\Spec}{Spec}
\DeclareMathOperator{\Sym}{Sym}
\DeclareMathOperator{\pf}{pf}
\DeclareMathOperator{\Ortho}{O}
\DeclareMathOperator{\diam}{diam}
\DeclareMathOperator{\Real}{Re}
\DeclareMathOperator{\Imag}{Im}
\DeclareMathOperator{\Arg}{Arg}

\newcommand{\C}{\mathbb{C}}
\newcommand{\R}{\mathbb{R}}
\newcommand{\Z}{\mathbb{Z}}
\newcommand{\N}{\mathbb{N}}


\DeclareMathOperator{\sla}{\mathfrak{sl}}
\newcommand{\norm}[1]{\left\lVert#1\right\rVert}
\newcommand{\transpose}{\intercal}

\newcommand{\conj}[1]{\bar{#1}}
\newcommand{\abs}[1]{\left|#1\right|}

%%% My commands, for solutions %%%

\DeclareMathOperator{\Log}{Log}
% To write df/(dx), use \pfrac{f}{x}
\newcommand{\pfrac}[2]{\frac{\partial #1}{\partial #2}}
% Partial derivative. To take d^2f/(dxdy), use \ppfrac[y]{f}{x}
% To take d^2f/(dx^2), use ppfrac{f}{x}
\newcommand{\ppfrac}[3][]{\frac{\partial^2 #2}{\ifthenelse{\isempty{#1}}{\partial #3^2}{\partial #3\partial #1}}}
 \newenvironment{solution}
  {\renewcommand\qedsymbol{$\blacksquare$}\begin{proof}[Solution]}
  {\end{proof}}






\usepackage{hyperref}
\author{Jim Fowler}
\course{Math 5522H}
\title{Assigned Readings}
\date{Week 4}

\usepackage{draftwatermark}
\SetWatermarkText{Draft}
\SetWatermarkScale{5}

\begin{document}
\maketitle

% Cauchy's theorem in a disk

From Palka's \textit{An Introduction to Complex Function Theory}, read
\begin{itemize}
\item V.1.1 Cauchy's Theorem For Rectangles
\item V.1.2 Integrals and Primitives
\item V.1.3 The Local Cauchy Theorem
\item V.2.1 Winding Numbers
\item V.2.2 Oriented Paths, Jordan Contours
\item V.2.3 The Local Integral Formula
\item V.3.1 Analyticity of Derivatives
\item V.3.2 Derivative Estimates
\item V.3.3 The Maximum Principle
\end{itemize}

From Stein and Shakarchi's \textit{Complex Analysis}, read
\begin{itemize}
\item 2.1 Goursat's theorem
\item 2.2 Local existence of primitives and Cauchy's theorem in a disc
\end{itemize}

From Ahlfors' \textit{Complex Analysis: An Introduction to The Theory of Analytic Functions of One Complex Variable}, read
\begin{itemize}
\item 4.1.4 Cauchy's Theorem for a Rectangle
\item 4.1.5 Cauchy's Theorem in a Disk
\end{itemize}
\end{document}
