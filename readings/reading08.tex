\documentclass{homework}
\input{preamble}
\usepackage{hyperref}
\author{Jim Fowler}
\course{Math 5522H}
\title{Assigned Readings}
\date{Week 8}

\begin{document}
\maketitle

Last week was cut short with the instructional break, so we took a
brief hiatus by looking at harmonic functions.  At the very end, we
were looking at functions harmonic in a punctured disk, and saw our
first glimpse of \textbf{removable singularities}.  Buoyed by the
harmonic story, we now apply what we learned to prove the
\textbf{Riemann extension theorem} for removable singularities\ldots
and more generally, this week we confront other sorts of
singularities, like \textbf{poles} and \textbf{essential
  singularities}.

There is quite a bit of material to work through, but we have already
seen some of these ideas (like poles) from a superficial point of
view, so a goal this week is to systematize our study of
singularities.  From Palka's \textit{An Introduction to Complex
  Function Theory}, read
\begin{itemize}
\item VIII.1.1 The Factor Theorem for Analytic Functions
\item VIII.1.2 Multiplicity
\item VIII.1.3 Discrete Sets, Discrete Mappings
\item VIII.2.1 Definition and Classification of Isolated Singularities
\item VIII.2.2 Removable Singularities
\item VIII.2.3 Poles
\item VIII.2.4 Meromorphic Functions
\item VIII.2.5 Essential Singularities
\item VIII.2.6 Isolated Singularities at Infinity
\item VIII.4.1 The Extended Complex Plane
\item VIII.4.2 The Extended Plane and Stereographic Projection
\item VIII.4.3 Functions in the Extended Setting
\item VIII.4.4 Topology in the Extended Plane
\item VIII.4.5 Meromorphic Functions and the Extended Plane
\end{itemize}

For a terser presentation, consult Stein and Shakarchi's \textit{Complex Analysis} by focusing on
\begin{itemize}
\item 3.1 Zeros and poles
\item 3.3 Singularities and meromorphic functions
\end{itemize}
or similarly consult Ahlfors' \textit{Complex Analysis: An Introduction to The Theory of Analytic Functions of One Complex Variable} by focusing on
\begin{itemize}
\item 4.3 Local Properties of Analytical Functions
\item 4.3.1 Removable Singularities. Taylor's Theorem
\item 4.3.2 Zeros and Poles
\item 4.3.3 The Local Mapping
\end{itemize}
As always, if you have questions or concerns about the speed of the
course, or if you have specific results that you want to make sure we
get to, please let me know!

Depending on the topic, the emphasis on the homework can vary quite a
bit.  For this week, you will see that the terminology section is a
bit longer than usual---there is a \textit{lot} of ``naming'' to be
done with the various sorts of singularities.  But next week, we'll be
learning about the \textbf{residue caclulus} and consequently will be
working on more numericals.

\end{document}
