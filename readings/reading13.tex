\documentclass{homework}
\input{preamble}
\usepackage{hyperref}
\author{Jim Fowler}
\course{Math 5522H}
\title{Assigned Readings}
\date{Week 13}

\begin{document}
\maketitle

We've seen many highlights thus far in the course, from general
machinery like Cauchy's theorem (wait, isn't every result just
Cauchy's theorem?) to beautiful special cases like our investigations
into the Gamma function.  But this week, we meet among the most
celebrated results in complex analysis, the \textbf{Riemann mapping
  theorem}, first formulated in Riemann's PhD thesis.

This amazing result states that, for any non-empty simply-connected
open subset $U$ of $\mathbb{C}$, either $U$ is all of $\mathbb{C}$ or there is
holomorphic map $U \to B_1(0)$ with holomorphic inverse.  Imagine how
complicated an arbitrary simply-connected subset of $\mathbb{C}$ can be, and
then imagine somehow being able to throw that complicated subset onto
the unit ball!  Amazing!

With such a significant result, all of our textbook references include
a discussion of the Riemann mapping theorem.  From Palka's \textit{An
  Introduction to Complex Function Theory}, read
\begin{itemize}
\item IX.3.1 Preparations
\item IX.3.2 The Mapping Theorem
\item IX.4.1 Topological Preliminaries
\item IX.4.2 Double Integrals
\item IX.4.3 Conformal Modulus
\item IX.4.4 Extending Conformal Mappings of the Unit Disk
\item IX.4.5 Jordan Domains
\item IX.4.6 Oriented Boundaries
\end{itemize}
or from Stein and Shakarchi's \textit{Complex Analysis}, read
\begin{itemize}
\item 8.3 The Riemann mapping theorem
\item 8.3.1 Necessary conditions and statement of the theorem
\item 8.3.2 Montel's theorem
\item 8.3.3 Proof of the Riemann mapping theorem
\end{itemize}
or from Ahlfors' \textit{Complex Analysis: An Introduction to The Theory of Analytic Functions of One Complex Variable}, read
\begin{itemize}
\item 6.1 The Riemann Mapping Theorem
\item 6.1.1 Statement and Proof
\item 6.1.2 Boundary Behavior
\item 6.1.3 Use of the Reflection Principle
\item 6.1.4 Analytic Arcs
\end{itemize}

During our time together, we have often had cause to think in terms of
M\"obius transformations.  Rember all those Blaschke factor problems!
We also have seen conformal maps and cross-ratio reaching all the way
back to the start of our time together.  Nevertheless, this week,
presents an opportunity to consolidate our understanding of these
topics, so feel free to consult Palka to review
\begin{itemize}
\item IX.1.1 Curvilinear Angles
\item IX.1.2 Diffeomorphisms
\item IX.1.3 Conformal Mappings
\item IX.1.4 Some Standard Conformal Mappings
\item IX.1.5 Self-Mappings of the Plane and Unit Disk
\item IX.1.6 Conformal Mappings in the Extended Plane
\item IX.2.1 Elementary M\"obius Transformations
\item IX.2.2 M\"obius Transformations and Matrices
\item IX.2.3 Fixed Points
\item IX.2.4 Cross-ratios
\item IX.2.5 Circles in the Extended Plane
\item IX.2.6 Reflection and Symmetry
\item IX.2.7 Classification of M\"obius Transformations
\item IX.2.8 Invariant Circles
\end{itemize}
Some similar (briefer!) content appear in Stein and Shakarchi's text, namely
\begin{itemize}
\item 8.1 Conformal equivalence and examples
\item 8.1.1 The disc and upper half-plane
\item 8.1.2 Further examples
\item 8.1.3 The Dirichlet problem in a strip
\item 8.2 The Schwarz lemma; automorphisms of the disc and upper half-plane
\item 8.2.1 Automorphisms of the disc
\item 8.2.2 Automorphisms of the upper half-plane
\end{itemize}

\end{document}
