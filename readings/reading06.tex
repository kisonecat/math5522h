\documentclass{homework}
\input{preamble}
\usepackage{hyperref}
\author{Jim Fowler}
\course{Math 5522H}
\title{Assigned Readings}
\date{Week 6}

\begin{document}
\maketitle

Having worked very hard for more than a month, we are ready to reap
our rewards and see the great consequences of Cauchy's theorem.  From
Palka's \textit{An Introduction to Complex Function Theory}, pay
attention to
\begin{itemize}
\item V.3.1 Analyticity of Derivatives
\item V.3.2 Derivative Estimates
\item V.3.3 The Maximum Principle
\end{itemize}
and note this \textbf{maximum modulus principle}.  This appeared on
last week's problem set, and this week we will unwrap significant
consequences of this principle, like the \textbf{Schwarz lemma}.

From Stein and Shakarchi's \textit{Complex Analysis}, read
\begin{itemize}
\item 2.5 Further applications
\item 2.5.1 Morera's theorem
\item 2.5.2 Sequences of holomorphic functions
\item 2.5.3 Holomorphic functions defined in terms of integrals
\item 2.5.4 Schwarz reflection principle
\item 2.5.5 Runge's approximation theorem
\end{itemize}
and pay special attention to \textbf{Morera's theorem} which, by
enabling us to study sequences of holomorphic functions, opens up new
methods for producing holomorphic functions.  The discussion of
Runge's approximation theorem should remind you of the technique we
used to tackle the general version of Cauchy's theorem.

Our texts perhaps do not emphasize the significance of Morera's
result.  If you recall our concept map from last week, we built much
of our work on Cauchy's theorem.  Morera is suggesting that we might
successfullyl study a ``larger'' class of functions which satisfy the
conclusion of Cauchy's theorem and nevertheless derive all our
favorite theorems---but that ``larger'' class turns out to be again
the holomorphic functions we know and love!

On the problem sets, you might be feeling bad that \textit{we just
  aren't doing enough integrals!}  Next week is Week 7, and then we'll
take a further detour to dig into harmonic functions (which is nicely
motivated by our experience with the maximum modulus principle).  But
then in Weeks~8 and 9 we will be back $\mathbb{C}$ and focused on
\textbf{residues} which will equip us to evaluate a great many
integrals.


\end{document}
