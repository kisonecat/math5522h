\documentclass{homework}
\input{preamble}
\usepackage{hyperref}
\author{Jim Fowler}
\course{Math 5522H}
\title{Assigned Readings}
\date{Week 7}

\begin{document}
\maketitle

If you've been feeling like we've been going too fast, good news: we
have a bit of a breather with the topic of \textbf{harmonic
  functions}.  Pedagogically, this week serves as a chance to catch-up
and reflect (!) on complex analysis and its relationship to the
``real'' case.

And with part of Spring Break having been moved into this week, the
homework this week is shorter, but I hope it is still enjoyable and
challenging.  The homework makes connections to previous topics like
harmonic conjugates, Wirtinger derivatives, maximum modulus principle.
And the problem about \textbf{universal Taylor series} is striking and
is sometimes referred to as \textbf{overconvergence}.

From Palka's \textit{An Introduction to Complex Function Theory}, read
\begin{itemize}
\item VI.1.1 Harmonic Conjugates
\item VI.2.1 The Mean Value Property
\item VI.2.2 Functions Harmonic in Annuli
\item VI.3.1 A Heat Flow Problem
\item VI.3.2 Poisson Integrals
\end{itemize}
Alternatively, from from Ahlfors' \textit{Complex Analysis}, read
\begin{itemize}
\item 4.6 Harmonic Functions
\item 4.6.1 Definition and Basic Properties
\item 4.6.2 The Mean-value Property
\item 4.6.3 Poisson's Formula
\item 4.6.4 Schwarz's Theorem
\item 4.6.5 The Reflection Principle
\end{itemize}
Importantly, note how tightly the story of harmonic functions is
related to the story of homomorphic functions.  Previous problem sets
have already highlighted some of these relationships (e.g., the
Poisson kernel), so this is a chance to consolidate our understanding.

As always, if you have questions or concerns about the course, our
pace, our goals, etc., please feel free to reach out.  Weeks 8 and 9
will mark a return to complex analysis proper, and build up to the
\textbf{calculus of residues} which will equip us to compute more
integrals!  There is always more to look forward to.  The homework
includes a computational question about the mysterious
\textbf{dilogarithm}, perhaps pointing towards more computations yet
to come.

\end{document}
