\documentclass{homework}
\usepackage{amsmath}
\DeclareMathOperator{\Mat}{Mat}
\DeclareMathOperator{\End}{End}
\DeclareMathOperator{\Hom}{Hom}
\DeclareMathOperator{\id}{id}
\DeclareMathOperator{\image}{im}
\DeclareMathOperator{\rank}{rank}
\DeclareMathOperator{\nullity}{nullity}
\DeclareMathOperator{\trace}{tr}
\DeclareMathOperator{\Spec}{Spec}
\DeclareMathOperator{\Sym}{Sym}
\DeclareMathOperator{\pf}{pf}
\DeclareMathOperator{\Ortho}{O}
\DeclareMathOperator{\diam}{diam}
\DeclareMathOperator{\Real}{Re}
\DeclareMathOperator{\Imag}{Im}
\DeclareMathOperator{\Arg}{Arg}

\newcommand{\C}{\mathbb{C}}
\newcommand{\R}{\mathbb{R}}
\newcommand{\Z}{\mathbb{Z}}
\newcommand{\N}{\mathbb{N}}


\DeclareMathOperator{\sla}{\mathfrak{sl}}
\newcommand{\norm}[1]{\left\lVert#1\right\rVert}
\newcommand{\transpose}{\intercal}

\newcommand{\conj}[1]{\bar{#1}}
\newcommand{\abs}[1]{\left|#1\right|}

%%% My commands, for solutions %%%

\DeclareMathOperator{\Log}{Log}
% To write df/(dx), use \pfrac{f}{x}
\newcommand{\pfrac}[2]{\frac{\partial #1}{\partial #2}}
% Partial derivative. To take d^2f/(dxdy), use \ppfrac[y]{f}{x}
% To take d^2f/(dx^2), use ppfrac{f}{x}
\newcommand{\ppfrac}[3][]{\frac{\partial^2 #2}{\ifthenelse{\isempty{#1}}{\partial #3^2}{\partial #3\partial #1}}}
 \newenvironment{solution}
  {\renewcommand\qedsymbol{$\blacksquare$}\begin{proof}[Solution]}
  {\end{proof}}






\usepackage{hyperref}
\author{Jim Fowler}
\course{Math 5522H}
\title{Assigned Readings}
\date{Week 10}

\begin{document}
\maketitle

The theme this week is \textbf{sequences and series of functions}
which provides an opportunity to consolidate our prior knowledge
(e.g., about Taylor series, about Laurent series), but also to push
forward with new terminology like \textbf{normal convergence} and
(time permitting) to new results like \textbf{Weierstrass infinite
  products}, but I expect much of this will have to wait.

To formalize our results around Taylor series and Laurent series, look
in Palka's \textit{An Introduction to Complex Function Theory} for
\begin{itemize}
\item VII.1.1 Uniform Convergence
\item VII.1.2 Normal Convergence
\item VII.2.1 Complex Series
\item VII.2.2 Series of Functions
\item VII.3.1 General Results
\item VII.3.2 Limit Superior of a Sequence
\item VII.3.3 Taylor Series
\item VII.3.4 Laurent Series
\end{itemize}
or take a look at Ahlfors' \textit{Complex Analysis}, focusing on
\begin{itemize}
\item 5.1 Power Series Expansions
\item 5.1.1 Wierstrass's Theorem
\item 5.1.2 The Taylor Series
\item 5.1.3 The Laurent Series
\end{itemize}
Recall that Laurent series showed up when we were considering isolated
singularities.  We will review \textbf{uniform convergence} and then
study \textbf{normal convergence} which, unsurprisingly given then
terrible overuse of the word ``normal'' goes by other names
(``pre-compact'' would be perfectly reasonable!).  Here
``normal'' doesn't mean ``usual'' but rather ``norm-al,'' i.e.,
relating to norms.  This culminates in \textbf{normal families} which
you can study from Palka's \textit{Complex Function Theory} in
\begin{itemize}  
\item VII.4.1 Normal Subfamilies of C(U)
\item VII.4.2 Equicontinuity
\item VII.4.3 The Arzela-Ascoli and Montel Theorems
\end{itemize}
I imagine you already saw Arzela-Ascoli in your real analysis course
(?), so Montel's theorem will provide an opportunity to connect to
your prior learning there.

Depending on time (and there's never enough), a new topic is
\textbf{infinite products} which you can find in Palka's
\textit{Complex Function Theory} in
\begin{itemize}  
\item X.2.1 Infinite Products
\item X.2.2 Infinite Products of Functions
\item X.2.3 Infinite Products and Analytic Functions
\end{itemize}
or in Stein and Shakarchi's \textit{Complex Analysis} in
\begin{itemize}
\item 5.1 Jensen's formula
\item 5.2 Functions of finite order
\item 5.3 Infinite products
\item 5.3.1 Generalities
\item 5.3.2 Example: the product formula for the sine function
\item 5.4 Weierstrass infinite products
\item 5.5 Hadamard's factorization theorem
\end{itemize}
and in Ahlfors' \textit{Complex Analysis} in
\begin{itemize}
\item 5.2 Partial Fractions and Factorization
\item 5.2.1 Partial Fractions
\item 5.2.2 Infinite Products
\item 5.2.3 Canonical Products
\item 5.2.4 The Gamma Function
\item 5.2.5 Stirling's Formula
\item 5.3 Entire Functions
\item 5.3.1 Jensen's Formula
\item 5.3.2 Hadamard's Theorem
\end{itemize}
I expect we'll just get started with infinite products.  There is a
lot of learning to do, but we have some flexibility with Week 12 to
return to these ideas.

\end{document}
