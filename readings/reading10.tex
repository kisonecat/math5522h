\documentclass{homework}
\usepackage{amsmath}
\DeclareMathOperator{\Mat}{Mat}
\DeclareMathOperator{\End}{End}
\DeclareMathOperator{\Hom}{Hom}
\DeclareMathOperator{\id}{id}
\DeclareMathOperator{\image}{im}
\DeclareMathOperator{\rank}{rank}
\DeclareMathOperator{\nullity}{nullity}
\DeclareMathOperator{\trace}{tr}
\DeclareMathOperator{\Spec}{Spec}
\DeclareMathOperator{\Sym}{Sym}
\DeclareMathOperator{\pf}{pf}
\DeclareMathOperator{\Ortho}{O}
\DeclareMathOperator{\diam}{diam}
\DeclareMathOperator{\Real}{Re}
\DeclareMathOperator{\Imag}{Im}
\DeclareMathOperator{\Arg}{Arg}

\newcommand{\C}{\mathbb{C}}
\newcommand{\R}{\mathbb{R}}
\newcommand{\Z}{\mathbb{Z}}
\newcommand{\N}{\mathbb{N}}


\DeclareMathOperator{\sla}{\mathfrak{sl}}
\newcommand{\norm}[1]{\left\lVert#1\right\rVert}
\newcommand{\transpose}{\intercal}

\newcommand{\conj}[1]{\bar{#1}}
\newcommand{\abs}[1]{\left|#1\right|}

%%% My commands, for solutions %%%

\DeclareMathOperator{\Log}{Log}
% To write df/(dx), use \pfrac{f}{x}
\newcommand{\pfrac}[2]{\frac{\partial #1}{\partial #2}}
% Partial derivative. To take d^2f/(dxdy), use \ppfrac[y]{f}{x}
% To take d^2f/(dx^2), use ppfrac{f}{x}
\newcommand{\ppfrac}[3][]{\frac{\partial^2 #2}{\ifthenelse{\isempty{#1}}{\partial #3^2}{\partial #3\partial #1}}}
 \newenvironment{solution}
  {\renewcommand\qedsymbol{$\blacksquare$}\begin{proof}[Solution]}
  {\end{proof}}






\usepackage{hyperref}
\author{Jim Fowler}
\course{Math 5522H}
\title{Assigned Readings}
\date{Week 10}

\begin{document}
\maketitle

The theme this week is \textbf{sequences and series of functions}
which provides an opportunity to consolidate our prior knowledge
(e.g., about Taylor series, about Laurent series), but also to push
forward with new terminology like \textbf{normal convergence}.

To formalize our results around Taylor series and Laurent series, look
in Palka's \textit{An Introduction to Complex Function Theory} for
\begin{itemize}
\item VII.1.1 Uniform Convergence
\item VII.1.2 Normal Convergence
\item VII.2.1 Complex Series
\item VII.2.2 Series of Functions
\item VII.3.1 General Results
\item VII.3.2 Limit Superior of a Sequence
\item VII.3.3 Taylor Series
\item VII.3.4 Laurent Series
\end{itemize}
or take a look at Ahlfors' \textit{Complex Analysis}, focusing on
\begin{itemize}
\item 5.1 Power Series Expansions
\item 5.1.1 Wierstrass's Theorem
\item 5.1.2 The Taylor Series
\item 5.1.3 The Laurent Series
\end{itemize}
Recall that Laurent series showed up when we were considering isolated
singularities.  We will review \textbf{uniform convergence} and then
study \textbf{normal convergence} which, unsurprisingly given then
terrible overuse of the word ``normal'' goes by other names (e.g.,
wouldn't ``pre-compact'' be a better name than ``normal family''?).
This culminates in \textbf{normal families} which you can study from
Palka's \textit{Complex Function Theory} in
\begin{itemize}  
\item VII.4.1 Normal Subfamilies of C(U)
\item VII.4.2 Equicontinuity
\item VII.4.3 The Arzela-Ascoli and Montel Theorems
\end{itemize}
I imagine you already saw Arzela-Ascoli in your real analysis course
(?), so Montel's theorem will provide an opportunity to connect to
your prior learning there.


\end{document}
