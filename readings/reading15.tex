\documentclass{homework}
\input{preamble}
\usepackage{hyperref}
\author{Jim Fowler}
\course{Math 5522H}
\title{Assigned Readings}
\date{Week 15}

\usepackage{draftwatermark}
\SetWatermarkText{Draft}
\SetWatermarkScale{5}

\begin{document}
\maketitle

We have been through so much together this term: beginning with an
exploration just what complex numbers are, we've met holomorphic
functions, harmonic functions, Cauchy's theorem and all its variants,
residues, singularities, line integrals, the Riemann mapping theorem,
infinite product formulas.

This week provides a capstone to those experiences, pointing to the
road ahead in a couple ways.  First, there are some connections to
other areas of mathematics (like hyperbolic geometry) yet to be made,
so we'll begin the week with a brief survey of those connections.  One
surprise is just how frequently complex analysis is employed in other
contexts, and it can be surprising to see complex analysis popping up
in the context of, say, number theory.

But the main goal---and what will occupy us for the latter half of the
week---is to dig deeper into ``analytic continuation,'' which is an
idea we've seen a bit before but now we'll take our exploration
deeper.  Take a look at Palka's \textit{An Introduction to Complex
  Function Theory} and read
\begin{itemize}
\item X.3.1 Extending Functions by Means of Taylor Series 
\item X.3.2 Analytic Continuation
\item X.3.3 Analytic Continuation Along Paths
\item X.3.4 Analytic Continuation and Homotopy
\end{itemize}
This discussion connects back to the Schwarz reflection principle,
which played a big role in our study of Schwarz-Christoffel maps last
week.  And similar ideas are discussed in Ahlfors' \textit{Complex
  Analysis}, for example in sections
\begin{itemize}
\item 8.1 Analytic Continuation
\item 8.1.1 The Weierstrass Theory
\item 8.1.2 Germs and Sheaves
\item 8.1.3 Sections and Riemann Surfaces
\item 8.1.4 Analytic Continuations along Arcs
\item 8.1.5 Homotopic Curves
\item 8.1.6 The Monodromy Theorem
\item 8.1.7 Branch Points
\end{itemize}
All of this points towards a study of Riemann surfaces.  A motivating
picture of such surfaces are the ``parking garages'' we sketched at
the very start of the course.  What we call ``branches'' are nothing
but levels in the parking garage.

Note that \textbf{there is no homework set this week}.  This is
intentional: many will need time to complete the remaining problem
sets.  Please make sure to submit any remaining problem sets.


\end{document}
