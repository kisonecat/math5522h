\documentclass{homework}
\usepackage{amsmath}
\DeclareMathOperator{\Mat}{Mat}
\DeclareMathOperator{\End}{End}
\DeclareMathOperator{\Hom}{Hom}
\DeclareMathOperator{\id}{id}
\DeclareMathOperator{\image}{im}
\DeclareMathOperator{\rank}{rank}
\DeclareMathOperator{\nullity}{nullity}
\DeclareMathOperator{\trace}{tr}
\DeclareMathOperator{\Spec}{Spec}
\DeclareMathOperator{\Sym}{Sym}
\DeclareMathOperator{\pf}{pf}
\DeclareMathOperator{\Ortho}{O}
\DeclareMathOperator{\diam}{diam}
\DeclareMathOperator{\Real}{Re}
\DeclareMathOperator{\Imag}{Im}
\DeclareMathOperator{\Arg}{Arg}

\newcommand{\C}{\mathbb{C}}
\newcommand{\R}{\mathbb{R}}
\newcommand{\Z}{\mathbb{Z}}
\newcommand{\N}{\mathbb{N}}


\DeclareMathOperator{\sla}{\mathfrak{sl}}
\newcommand{\norm}[1]{\left\lVert#1\right\rVert}
\newcommand{\transpose}{\intercal}

\newcommand{\conj}[1]{\bar{#1}}
\newcommand{\abs}[1]{\left|#1\right|}

%%% My commands, for solutions %%%

\DeclareMathOperator{\Log}{Log}
% To write df/(dx), use \pfrac{f}{x}
\newcommand{\pfrac}[2]{\frac{\partial #1}{\partial #2}}
% Partial derivative. To take d^2f/(dxdy), use \ppfrac[y]{f}{x}
% To take d^2f/(dx^2), use ppfrac{f}{x}
\newcommand{\ppfrac}[3][]{\frac{\partial^2 #2}{\ifthenelse{\isempty{#1}}{\partial #3^2}{\partial #3\partial #1}}}
 \newenvironment{solution}
  {\renewcommand\qedsymbol{$\blacksquare$}\begin{proof}[Solution]}
  {\end{proof}}






\usepackage{hyperref}
\author{Jim Fowler}
\course{Math 5522H}
\title{Assigned Readings}
\date{Week 14}

\begin{document}
\maketitle

Sadly, this week introduces our final homework assignment; in the
words of Vi Hart, there
\href{https://www.youtube.com/watch?v=UkSN-kqAmxw}{``ain't more thing
  to climb.''}  The pedagogical reason for this being the final
homework is so that you can spend the last week of the course
consolidating and preparing for the last quiz, which plays the role of
the final examination in this 100\% online course.  Make the most of
the time!  And please submit homework that you have not yet submitted
if you are behind.

The primary theme this week is \textbf{conformal maps onto polygons}.  Towards that end, take a look at
Palka's \textit{An Introduction to Complex Function Theory} and read
\begin{itemize}
\item IX.5.1 Polygons
\item IX.5.2 The Reflection Principle
\item IX.5.3 The Schwarz-Christoffel Formula
\end{itemize}
Given the challenge that we had last week in custom-building conformal
maps, the ease with which the Schwarz-Christoffel formula produces out
so many nice examples should impress you!  This is such a classic topic that it is covered in all our textbooks; if you want a different presentation, consult Stein and Shakarchi's \textit{Complex Analysis} in
\begin{itemize}
\item 8.4 Conformal mappings onto polygons
\item 8.4.1 Some examples
\item 8.4.2 The Schwarz-Christoffel integral
\item 8.4.3 Boundary behavior
\item 8.4.4 The mapping formula
\end{itemize}
or Ahlfors' \textit{Complex Analysis} and read
\begin{itemize}
\item 6.2 Conformal Mapping of Polygons
\item 6.2.1 The Behavior at an Angle
\item 6.2.2 The Schwarz-Christoffel Formula
\item 6.2.3 Mapping on a Rectangle
\item 6.2.4 The Triangle Functions of Schwarz
\end{itemize}
There is much more that remains to be done besides conformal maps onto
polygons.  To give a taste of what lies ahead, we'll sample a few
other topics as time permits before summarizing the course next week
through the lens of Riemann surfaces.


\end{document}
