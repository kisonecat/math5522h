\documentclass{homework}
\usepackage{amsmath}
\DeclareMathOperator{\Mat}{Mat}
\DeclareMathOperator{\End}{End}
\DeclareMathOperator{\Hom}{Hom}
\DeclareMathOperator{\id}{id}
\DeclareMathOperator{\image}{im}
\DeclareMathOperator{\rank}{rank}
\DeclareMathOperator{\nullity}{nullity}
\DeclareMathOperator{\trace}{tr}
\DeclareMathOperator{\Spec}{Spec}
\DeclareMathOperator{\Sym}{Sym}
\DeclareMathOperator{\pf}{pf}
\DeclareMathOperator{\Ortho}{O}
\DeclareMathOperator{\diam}{diam}
\DeclareMathOperator{\Real}{Re}
\DeclareMathOperator{\Imag}{Im}
\DeclareMathOperator{\Arg}{Arg}

\newcommand{\C}{\mathbb{C}}
\newcommand{\R}{\mathbb{R}}
\newcommand{\Z}{\mathbb{Z}}
\newcommand{\N}{\mathbb{N}}


\DeclareMathOperator{\sla}{\mathfrak{sl}}
\newcommand{\norm}[1]{\left\lVert#1\right\rVert}
\newcommand{\transpose}{\intercal}

\newcommand{\conj}[1]{\bar{#1}}
\newcommand{\abs}[1]{\left|#1\right|}

%%% My commands, for solutions %%%

\DeclareMathOperator{\Log}{Log}
% To write df/(dx), use \pfrac{f}{x}
\newcommand{\pfrac}[2]{\frac{\partial #1}{\partial #2}}
% Partial derivative. To take d^2f/(dxdy), use \ppfrac[y]{f}{x}
% To take d^2f/(dx^2), use ppfrac{f}{x}
\newcommand{\ppfrac}[3][]{\frac{\partial^2 #2}{\ifthenelse{\isempty{#1}}{\partial #3^2}{\partial #3\partial #1}}}
 \newenvironment{solution}
  {\renewcommand\qedsymbol{$\blacksquare$}\begin{proof}[Solution]}
  {\end{proof}}






\usepackage{hyperref}
\author{Jim Fowler}
\course{Math 5522H}
\title{Assigned Readings}
\date{Week 12}

\begin{document}
\maketitle

\textit{This is a short week with an instructional break, so the
  problem set is shorter.}

When studying special functions, we met various product formulas along
the way.  From those special cases, we desire general theorems, so
this week we study infinite products in some generality, and in
particular we study \textbf{Weierstrass infinite products} letting us
write an entire function as an infinite products involving its zeros.
This should feel like a great generalization of the fundamental
theorem of algebra, completing a circle of ideas that we began at the
start of our time together.

For the readings, turn to Palka's
\textit{Complex Function Theory} in
\begin{itemize}  
\item X.2.1 Infinite Products
\item X.2.2 Infinite Products of Functions
\item X.2.3 Infinite Products and Analytic Functions
\end{itemize}
or in Stein and Shakarchi's \textit{Complex Analysis} in
\begin{itemize}
\item 5.1 Jensen's formula
\item 5.2 Functions of finite order
\item 5.3 Infinite products
\item 5.3.1 Generalities
\item 5.3.2 Example: the product formula for the sine function
\item 5.4 Weierstrass infinite products
\item 5.5 Hadamard's factorization theorem
\end{itemize}
or to Ahlfors' \textit{Complex Analysis} in
\begin{itemize}
\item 5.2 Partial Fractions and Factorization
\item 5.2.1 Partial Fractions
\item 5.2.2 Infinite Products
\item 5.2.3 Canonical Products
\item 5.3 Entire Functions
\item 5.3.1 Jensen's Formula
\item 5.3.2 Hadamard's Theorem
\end{itemize}

As you read, I hope you'll reflect on a principle we've begun to see:
entire functions can often be studied via singularities (which, for
us, means poles and zeros) and via their growth rates.  We saw some of
this when we explored the Gamma function last week---the proof of
Euler's reflection theorem in lecture relied on growth rate estimates,
an argument we reprised when we studied an infinite series for
cotangent.  This leads to our defining what it means for a function to
be of \textbf{finite order} and finally to \textbf{Hadamard's
  refinement} of Weierstrass' factorization.

\end{document}
