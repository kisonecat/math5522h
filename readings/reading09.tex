\documentclass{homework}
\input{preamble}
\usepackage{hyperref}
\author{Jim Fowler}
\course{Math 5522H}
\title{Assigned Readings}
\date{Week 9}

\begin{document}
\maketitle

After many weeks of ``theory'' it is time to reap the reward of
computing a ton of integrals.  Each of the recommended textbooks
address \textbf{residues}.  For example, take a look at Palka's
\textit{An Introduction to Complex Function Theory} and read
\begin{itemize}
\item VIII.3.1 The Residue Theorem
\item VIII.3.2 Evaluating Integrals with the Residue Theorem
\item VIII.3.3 Consequences of the Residue Theorem
\end{itemize}
or from Stein and Shakarchi's \textit{Complex Analysis} read
\begin{itemize}
\item 3.2 The residue formula
\item 3.2.1 Examples
\item 3.4 The argument principle and applications
\item 3.5 Homotopies and simply connected domains
\item 3.6 The complex logarithm
\end{itemize}
or from Ahlfors' \textit{Complex Analysis}, read
\begin{itemize}
\item 4.5 The Calculus of Residues
\item 4.5.1 The Residue Theorem
\item 4.5.2 The Argument Principle
\item 4.5.3 Evaluation of Definite Integrals
\end{itemize}
Last week, we first discussed isolated singularities and this week, we
will focus on residues, which encode information about the contour
integral around a pole.  The main surprise is that it isn't especially
difficult to compute these residues, and by developing our creativity
with contours, we can effectively evaluate a number of otherwise
challenging integrals.

The homework this week is pretty heavy on integrals, but I hope you
will see past the individual examples and develop some intuition for
the common patterns.  An example of this is \textbf{Jordan's lemma}
from Week 8.

\end{document}
