\documentclass{homework}
\usepackage{amsmath}
\DeclareMathOperator{\Mat}{Mat}
\DeclareMathOperator{\End}{End}
\DeclareMathOperator{\Hom}{Hom}
\DeclareMathOperator{\id}{id}
\DeclareMathOperator{\image}{im}
\DeclareMathOperator{\rank}{rank}
\DeclareMathOperator{\nullity}{nullity}
\DeclareMathOperator{\trace}{tr}
\DeclareMathOperator{\Spec}{Spec}
\DeclareMathOperator{\Sym}{Sym}
\DeclareMathOperator{\pf}{pf}
\DeclareMathOperator{\Ortho}{O}
\DeclareMathOperator{\diam}{diam}
\DeclareMathOperator{\Real}{Re}
\DeclareMathOperator{\Imag}{Im}
\DeclareMathOperator{\Arg}{Arg}

\newcommand{\C}{\mathbb{C}}
\newcommand{\R}{\mathbb{R}}
\newcommand{\Z}{\mathbb{Z}}
\newcommand{\N}{\mathbb{N}}


\DeclareMathOperator{\sla}{\mathfrak{sl}}
\newcommand{\norm}[1]{\left\lVert#1\right\rVert}
\newcommand{\transpose}{\intercal}

\newcommand{\conj}[1]{\bar{#1}}
\newcommand{\abs}[1]{\left|#1\right|}

%%% My commands, for solutions %%%

\DeclareMathOperator{\Log}{Log}
% To write df/(dx), use \pfrac{f}{x}
\newcommand{\pfrac}[2]{\frac{\partial #1}{\partial #2}}
% Partial derivative. To take d^2f/(dxdy), use \ppfrac[y]{f}{x}
% To take d^2f/(dx^2), use ppfrac{f}{x}
\newcommand{\ppfrac}[3][]{\frac{\partial^2 #2}{\ifthenelse{\isempty{#1}}{\partial #3^2}{\partial #3\partial #1}}}
 \newenvironment{solution}
  {\renewcommand\qedsymbol{$\blacksquare$}\begin{proof}[Solution]}
  {\end{proof}}






\usepackage{hyperref}
\author{Jim Fowler}
\course{Math 5522H}
\title{Assigned Readings}
\date{Week 9}

\begin{document}
\maketitle

After many weeks of ``theory'' it is time to reap the reward of
computing a ton of integrals.  Each of the recommended textbooks
address \textbf{residues}.  For example, take a look at Palka's
\textit{An Introduction to Complex Function Theory} and read
\begin{itemize}
\item VIII.3.1 The Residue Theorem
\item VIII.3.2 Evaluating Integrals with the Residue Theorem
\item VIII.3.3 Consequences of the Residue Theorem
\end{itemize}
or from Stein and Shakarchi's \textit{Complex Analysis} read
\begin{itemize}
\item 3.2 The residue formula
\item 3.2.1 Examples
\item 3.4 The argument principle and applications
\item 3.5 Homotopies and simply connected domains
\item 3.6 The complex logarithm
\end{itemize}
or from Ahlfors' \textit{Complex Analysis}, read
\begin{itemize}
\item 4.5 The Calculus of Residues
\item 4.5.1 The Residue Theorem
\item 4.5.2 The Argument Principle
\item 4.5.3 Evaluation of Definite Integrals
\end{itemize}
Last week, we first discussed isolated singularities and this week, we
will focus on residues, which encode information about the contour
integral around a pole.  The main surprise is that it isn't especially
difficult to compute these residues, and by developing our creativity
with contours, we can effectively evaluate a number of otherwise
challenging integrals.

The homework this week is pretty heavy on integrals, but I hope you
will see past the individual examples and develop some intuition for
the common patterns.  An example of this is \textbf{Jordan's lemma}
from Week 8.

\end{document}
