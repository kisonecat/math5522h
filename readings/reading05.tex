\documentclass{homework}
\usepackage{amsmath}
\DeclareMathOperator{\Mat}{Mat}
\DeclareMathOperator{\End}{End}
\DeclareMathOperator{\Hom}{Hom}
\DeclareMathOperator{\id}{id}
\DeclareMathOperator{\image}{im}
\DeclareMathOperator{\rank}{rank}
\DeclareMathOperator{\nullity}{nullity}
\DeclareMathOperator{\trace}{tr}
\DeclareMathOperator{\Spec}{Spec}
\DeclareMathOperator{\Sym}{Sym}
\DeclareMathOperator{\pf}{pf}
\DeclareMathOperator{\Ortho}{O}
\DeclareMathOperator{\diam}{diam}
\DeclareMathOperator{\Real}{Re}
\DeclareMathOperator{\Imag}{Im}
\DeclareMathOperator{\Arg}{Arg}

\newcommand{\C}{\mathbb{C}}
\newcommand{\R}{\mathbb{R}}
\newcommand{\Z}{\mathbb{Z}}
\newcommand{\N}{\mathbb{N}}


\DeclareMathOperator{\sla}{\mathfrak{sl}}
\newcommand{\norm}[1]{\left\lVert#1\right\rVert}
\newcommand{\transpose}{\intercal}

\newcommand{\conj}[1]{\bar{#1}}
\newcommand{\abs}[1]{\left|#1\right|}

%%% My commands, for solutions %%%

\DeclareMathOperator{\Log}{Log}
% To write df/(dx), use \pfrac{f}{x}
\newcommand{\pfrac}[2]{\frac{\partial #1}{\partial #2}}
% Partial derivative. To take d^2f/(dxdy), use \ppfrac[y]{f}{x}
% To take d^2f/(dx^2), use ppfrac{f}{x}
\newcommand{\ppfrac}[3][]{\frac{\partial^2 #2}{\ifthenelse{\isempty{#1}}{\partial #3^2}{\partial #3\partial #1}}}
 \newenvironment{solution}
  {\renewcommand\qedsymbol{$\blacksquare$}\begin{proof}[Solution]}
  {\end{proof}}






\usepackage{hyperref}
\author{Jim Fowler}
\course{Math 5522H}
\title{Assigned Readings}
\date{Week 5}

\usepackage{draftwatermark}
\SetWatermarkText{Draft}
\SetWatermarkScale{5}

\begin{document}
\maketitle

% Cauchy's integral formula

From Palka's \textit{An Introduction to Complex Function Theory}, read
\begin{itemize}
\item V.4.1 Branches of Logarithms of Functions
\item V.4.2 Logarithms of Rational Functions
\item V.4.3 Branches of Powers of Functions
\item V.5.1 Iterated Line Integrals
\item V.5.2 Cycles
\item V.5.3 Cauchy's Theorem and Integral Formula
\item V.6.1 Simply Connected Domains
\item V.6.2 Simple Connectivity, Primitives, and Logarithms
\item V.7.1 Homotopic Paths
\item V.7.2 Contractible Paths
\end{itemize}

From Stein and Shakarchi's \textit{Complex Analysis}, read
\begin{itemize}
\item 2.3 Evaluation of some integrals
\item 2.4 Cauchy's integral formulas
\end{itemize}

From Ahlfors' \textit{Complex Analysis: An Introduction to The Theory of Analytic Functions of One Complex Variable}, read
\begin{itemize}
\item 4.2 Cauchy's Integral Formula
\item 4.2.1 The Index of a Point with Respect to a Closed Curve
\item 4.2.2 The Integral Formula
\item 4.2.3 Higher Derivatives
\item 4.4 The General Form of Cauchy's Theorem
\item 4.4.1 Chains and Cycles
\item 4.4.2 Simple Connectivity
\item 4.4.3 Homology
\item 4.4.4 The General Statement of Cauchy's Theorem
\item 4.4.5 Proof of Cauchy's Theorem
\end{itemize}
\end{document}
