\documentclass{homework}
\input{preamble}
\usepackage{hyperref}
\author{Jim Fowler}
\course{Math 5522H}
\title{Assigned Readings}
\date{Week 11}

\usepackage{draftwatermark}
\SetWatermarkText{Draft}
\SetWatermarkScale{5}

\begin{document}
\maketitle

You've worked so hard this term, so it is important that you take time
to reap what you've planted, important that we work through examples
of the machinery we've built so carefully.

Towards that end, this week is more of an ``application'' or an
``example'' week, focused around special functions.  Our focus is
specifically on the \textbf{Gamma function} and the \textbf{zeta
  function.}  Remember that Gamma is written $\Gamma$ and zeta is
written $\zeta$.  Our brief foray into these special functions will
motivate much of what follows, e.g., having seen some significant
examples of infinite products will motivate next week's deeper dive
into topics like Weierstrass' factorization theorem.

For this week, you could look at Palka's Section~X.2.4, which
addresses the Gamma function.  But I think it is wiser to focus on our
other two references.  From Stein and Shakarchi's \textit{Complex
  Analysis}, read
\begin{itemize}
\item 6.1 The Gamma function  
\item 6.2 The zeta function
\item 6.2.1 Functional equation and analytic continuation
\item 7.1 Zeros of the zeta function
\item 7.1.1 Estimates
\end{itemize}
Note that ``analytic continuation'' appears, despite our not having
talked about analytic continuation in general.  This is intentional:
in learning mathematics, sometimes we start from general principles
and consider special cases, and other times it is more natural to
begin with a special case and then discover the general principle.
Thus far, we've often been considering \textit{entire} functions so we
haven't necessarily had to worry about continuation.  For the Riemann
zeta function, the definition is given as
\[
  \zeta (s)=\sum_{n=1}^\infty \frac {1}{n^{s}}
\]
but note that this series doesn't converge everywhere; this converges
when the real part of $s$ is bigger than one.  So to compute say
$\zeta(0) = -1/2$, we need to find a meromorphic function agreeing
with the above series.

The function $\zeta$ is also discussed in Ahlfors' textbook, in
sections
\begin{itemize}
\item 5.4 The Riemann Zeta Function
\item 5.4.1 The Product Development
\item 5.4.2 Extension of Zeta to the Whole Plane
\item 5.4.3 The Functional Equation
\item 5.4.4 The Zeros of the Zeta Function
\end{itemize}


\end{document}
